\documentclass[twoside,english]{uiofysmaster}

\usepackage{preamble}
\usepackage{subfiles}
\usepackage{datetime}
\newdateformat{monthyeardate}{%
  \monthname[\THEMONTH], \THEYEAR}


\author{Bendik Samseth}
\title{(Title Tinkering Needed)Bootstrapping Variational Monte Carlo with Machine Learning}
\date{\monthyeardate\today}

\begin{document}

\maketitle

\begin{abstract}
Recent applications of machine learning for quantum mechanics have shown
encouraging results in efforts to overcome the exponential scaling complexity of
the many-body wave function. We continue this exploration by introducing a
neural network as an additional Jastrow factor to the wave function ansatz
within the framework of Variational Monte Carlo (VMC), with the hope of learning
correlations beyond what traditional methods have been able to.

We begin with a review of the relevant parts of quantum mechanics with particular
emphasis on Monte Carlo methods. We then introduce central elements of machine
learning, focusing on neural networks. Finally we discuss implementation
challenges.

We test our approach first on interacting electrons in a harmonic oscillator
potential. The energy estimates approach those of Diffusion Monte Carlo,
outperforming the accuracy of benchmarks by more than an order of magnitude,
with a similar lowering of the standard deviation.

Second we apply the same technique to the strongly correlated system of liquid
$^4$He. Again we obtain significant lowering of the ground state energy,
although the optimization problem proves far more challenging.

These improvements come at the cost of greatly increased computing time,
however, we argue that the time complexity can still be made to scale similarly to
$\mathcal{O}(N^2)$ with the number of particles.
\end{abstract}

\begin{acknowledgements}
  I acknowledge my acknowledgements.
\end{acknowledgements}

\tableofcontents

\subfile{Introduction.tex}

\part{Theory}
\label{prt:theory}
\subfile{QuantumTheory.tex}
\subfile{VariationalMonteCarlo.tex}
\subfile{MonteCarlo.tex}
\subfile{MachineLearning.tex}
\subfile{MergingVMCandML.tex}
\part{Implementation}
\label{prt:implementation}
\subfile{Parallelization.tex}
\subfile{AutomaticDifferentiation.tex}
\subfile{Qflow.tex}
\subfile{Verification.tex}
\part{Results}
\label{prt:results}
\subfile{QuantumDots.tex}
\subfile{LiquidHelium.tex}
\part{Conclusion and Outlook}
\label{prt:conclusion}
\subfile{Conclusion.tex}
\subfile{Appendix.tex}
\printbibliography[heading=bibintoc, title={References}]


\end{document}
