\documentclass[Thesis.tex]{subfiles}
\begin{document}
\chapter{Quantum Dots}
\label{chp:quantum-dots}

We now present results for all methods discussed, applied on the example system
of Quantum Dots (QD) from \cref{sec:quantum-dots-theory}. We present first some
benchmarks using a typical Slater-Jastrow wave function form, followed by
introduction of various neural network-based wave function.

\section{Benchmark}

As we shall restrict this analysis to only two interacting particles in the QD,
$\vX = (\vx_1\ \vx_2)$, our benchmark wave function is rather simple. We build
it up using the product of single particle ground states (Gaussian), multiplied
by a Pade-Jastrow correlation term\footnote{We drop the constant factor
  from~\cref{eq:Phi-non-inter} because we have not normalized the wave function.}:

\begin{align}
  \label{eq:qd-pade-jastrow-anzats}
  \psi_{PJ}(\vX) &= \Phi(\vX) \,J_P(\vX)\\
  &= \exp(-\alpha_G\sum_{i=1}^N \norm{\vx_i}^2 + \sum_{i < j} \frac{\alpha_{PJ}
    r_{ij}}{1 + \beta_{PJ} r_{ij}}),
\end{align}
where $\alpha_{PJ} = 1$ is fixed by the cusp conditions, and $\alpha_G$ and $\beta_{PJ}$
are the two only variational parameters.

\subsubsection{Optimizing}

We have run a simple optimization of the above wave function, using initial
values of $\alpha_G = 0.5$ and $\beta_{PJ} = 1$. We used importance sampling and the
ADAM optimization scheme. We used $\num{2000}$ optimization steps, each with
$\num{5000}$ MC cycles. The values are somewhat arbitrarily, and we get
similar results for several other choices.

\begin{figure}[h]
   \centering
    \resizebox{\linewidth}{!}{%
        % This file was created by matplotlib2tikz v0.7.4.
\begin{tikzpicture}

\definecolor{color0}{rgb}{0.12156862745098,0.466666666666667,0.705882352941177}
\definecolor{color1}{rgb}{1,0.498039215686275,0.0549019607843137}

\begin{groupplot}[group style={group size=2 by 1}]
\nextgroupplot[
legend cell align={left},
legend style={draw=white!80.0!black},
tick align=outside,
tick pos=left,
x grid style={white!69.01960784313725!black},
xlabel={\% of training},
ylabel={Ground state energy $[\si{\au}]$},
xmin=-5, xmax=105,
xtick style={color=black},
y grid style={white!69.01960784313725!black},
ymin=2.99704592819679, ymax=3.0553945154196,
ytick style={color=black},
ylabel near ticks,
]
\addplot [semithick, color0]
table {%
0 3.05274230690948
1 3.03793728796287
2 3.03042932078306
3 3.03147832063556
4 3.02844507193604
5 3.02088514786338
6 3.02511840035016
7 3.01862389645188
8 3.01922001846441
9 3.02448078551538
10 3.01695374152415
11 3.02149449163837
12 3.01751935807795
13 3.01271928794601
14 3.01647427586709
15 3.00939605941487
16 3.01386722338982
17 3.01304696248959
18 3.00992166949493
19 3.00892353607615
20 3.01016826114483
21 3.01039088846332
22 3.00426445580458
23 3.00645324289482
24 3.00501888357255
25 3.00348186918386
26 3.00441720089552
27 3.00442281302622
28 3.0018750604523
29 3.00355148586118
30 3.00171137956311
31 3.00298038895463
32 3.00080951818635
33 3.00105332554284
34 3.00195265921159
35 3.0005461422342
36 2.9999000956281
37 3.00019076693779
38 3.00076928146752
39 3.00042162249406
40 3.00135401658992
41 3.00090852610597
42 3.0007619299255
43 3.00072168045281
44 3.00064448372425
45 3.00040557287456
46 3.00111714273009
47 3.00076544340293
48 3.00035070540922
49 3.00092541277394
50 3.00022734746425
51 3.00046526568344
52 3.00059781409551
53 3.00063381420564
54 3.00035451805525
55 3.00037117495291
56 3.00134892692779
57 3.00075190853335
58 3.00065477638969
59 3.00021139892498
60 3.00029658491584
61 3.00049936563556
62 3.00049566724919
63 3.00063981769536
64 3.00062485027585
65 3.00080855104683
66 3.00035673435316
67 3.00068375879364
68 3.0008476157282
69 3.00092963123955
70 3.00071206112081
71 3.00038036745322
72 3.00059580814765
73 3.00064194367538
74 3.00047228782548
75 3.00051056917616
76 3.00018337005192
77 3.00031215610914
78 3.00084542377648
79 3.00138368193253
80 3.00130159778628
81 3.00073305314236
82 3.00025821935658
83 3.00029180410207
84 3.00022019576071
85 3.00053189012777
86 3.00050436447339
87 3.00075486546648
88 3.00072817806753
89 3.0011932402123
90 3.00098661658882
91 3.00107506850719
92 3.00025494563264
93 2.999880129195
94 2.99969813670692
95 3.00049469036751
96 3.00141903905299
97 3.00018861498403
98 3.00065423672549
99 3.0002256484428
100 3.000608740078
};
\addlegendentry{$\psi_{PJ}$}
\addplot [semithick, black, opacity=0.5, dashed]
table {%
-5 3
105 3
};
\addlegendentry{Exact}

\nextgroupplot[
legend cell align={left},
legend style={draw=white!80.0!black},
tick align=outside,
tick pos=left,
x grid style={white!69.01960784313725!black},
xlabel={\% of training},
ylabel={Parameters},
xmin=-5, xmax=105,
xtick style={color=black},
y grid style={white!69.01960784313725!black},
ymin=0.366969287310357, ymax=1.03014431965189,
ytick style={color=black},
ylabel near ticks,
ytick pos=right,
yticklabel pos=right,
ylabel style={rotate=-180},
]
\addplot [semithick, color0]
table {%
0 0.5
1 0.480655926967057
2 0.463697501976467
3 0.451192127077001
4 0.442737507443148
5 0.438111608232936
6 0.436809678602654
7 0.439550227846682
8 0.440370661362374
9 0.44196620588902
10 0.441699815490073
11 0.441673424505602
12 0.442794615290762
13 0.445288540681495
14 0.44524344435607
15 0.44599998860998
16 0.449872276110169
17 0.451396959301731
18 0.454260819395208
19 0.455092374589909
20 0.456843530760932
21 0.458709500679744
22 0.460615876660145
23 0.460374712476665
24 0.463261215859744
25 0.464949617594546
26 0.465163534320254
27 0.469088094313409
28 0.473001222172205
29 0.473886018788919
30 0.474397605060466
31 0.4766342480579
32 0.479277060460206
33 0.481446936562426
34 0.482341299014396
35 0.483129614027315
36 0.485187340832105
37 0.485484833442987
38 0.486849156774872
39 0.488539929624974
40 0.489226179672706
41 0.49061368339717
42 0.491287205504628
43 0.49121271170576
44 0.491618280745601
45 0.491856226891644
46 0.49209571237464
47 0.493312673786748
48 0.493344650527207
49 0.493349972324268
50 0.493554335750895
51 0.493742625998036
52 0.493684536935949
53 0.493418273715207
54 0.493663912511312
55 0.493795231166937
56 0.494313551308408
57 0.494756080546951
58 0.49396689406379
59 0.49461122060446
60 0.494502925522114
61 0.494288988911376
62 0.494628077475571
63 0.49413752152439
64 0.49414972531498
65 0.493971120843915
66 0.494164409824097
67 0.493759655306835
68 0.494003530172205
69 0.494028572113906
70 0.494729176763253
71 0.494287778501983
72 0.493450583197636
73 0.493995142627925
74 0.493989208145016
75 0.493880220631274
76 0.493849266353634
77 0.494615161941479
78 0.494383290905142
79 0.49392515784999
80 0.49490124712869
81 0.494709209883818
82 0.494705592833194
83 0.49453674993233
84 0.494268510488658
85 0.493701388063222
86 0.494591361060595
87 0.494108037162981
88 0.494689893738149
89 0.494466558744206
90 0.493822377138241
91 0.494849593582664
92 0.494919036209166
93 0.493659689615945
94 0.493698639179818
95 0.494660104397708
96 0.494601257652416
97 0.494625829393491
98 0.494284436275973
99 0.494189231282311
100 0.49501908671325
};
\addlegendentry{$\alpha_G$}
\addplot [semithick, color1]
table {%
0 1
1 0.980141471446477
2 0.961154099549839
3 0.94354541962038
4 0.926263381376864
5 0.909188306126108
6 0.89229994851214
7 0.875877528367407
8 0.859278833730977
9 0.842864550195302
10 0.825497233124107
11 0.807977785668522
12 0.791502880092063
13 0.774335581707413
14 0.756734655277644
15 0.739382782996785
16 0.722002653039462
17 0.704091868819612
18 0.685961062328723
19 0.668534931746746
20 0.65201323793735
21 0.634798789489699
22 0.618312733285203
23 0.601305182751218
24 0.58601777009998
25 0.571094569781092
26 0.556726229571795
27 0.543178962981069
28 0.529220850286612
29 0.515032109619336
30 0.502466196498837
31 0.49118036743747
32 0.480106406224218
33 0.469854110847922
34 0.459905925210542
35 0.45169073074283
36 0.444085166937463
37 0.436721490015063
38 0.43086027216831
39 0.426019836452593
40 0.421567023051503
41 0.418175341140745
42 0.415147180000562
43 0.412114996741048
44 0.409742178054753
45 0.407544904686301
46 0.406108097925919
47 0.405401962226926
48 0.404109574090475
49 0.403000552712037
50 0.402386691318389
51 0.40198936056923
52 0.401304016882003
53 0.400463395040488
54 0.399907486357164
55 0.399927548840752
56 0.399439318254555
57 0.399230167045008
58 0.398594625417332
59 0.398836667613015
60 0.398714339786569
61 0.398588601560595
62 0.398655699546475
63 0.397913251765501
64 0.397897219990408
65 0.397471495679413
66 0.397552186914199
67 0.39713574127287
68 0.397113606962245
69 0.397829064337513
70 0.399062510574475
71 0.398682882777356
72 0.397791930555549
73 0.398332439546336
74 0.39840688173924
75 0.398196740761694
76 0.397787173238738
77 0.397493872059313
78 0.398208183155796
79 0.397970236915008
80 0.398404448782546
81 0.398493589328846
82 0.399081438402194
83 0.398881611564127
84 0.398476913330119
85 0.398767471837898
86 0.398649655081354
87 0.398289170410177
88 0.399078939980803
89 0.398678567980949
90 0.39764844532161
91 0.397948711272731
92 0.399085817637872
93 0.398518195003331
94 0.39858533693385
95 0.398898472417307
96 0.398514097278833
97 0.397936245658474
98 0.398139502941347
99 0.397847090314031
100 0.397793881004863
};
\addlegendentry{$\beta_{PJ}$}
\end{groupplot}

\end{tikzpicture}
    }
    \caption{\label{fig:QD-benchmark-pade-jastrow-training}Left: Performance of the
      wave function in \cref{eq:qd-pade-jastrow-anzats} as a function of
      training steps. Right: Progression of variational parameters as a function
      of training steps. The source code for this graphic can be found~\cite[TODO: Add
    path]{MS-thesis-repository}, and \LaTeX{} output generated
    by~\cite{nico_schlomer_2018_1173090}.}
\end{figure}

\begin{table}[h]
  \centering
  \begin{tabular}{lccccc}
\toprule
\addlinespace
& $\langle E_L\rangle$ & CI$^{95}_-$ & CI$^{95}_+$ & Std & Var \\
\midrule
    $\Phi$ & 3.250(2) & 3.246 & 3.254 & \num{1.8e-01} & \num{3.3e-02}\\
$\psi_{PJ}$ & 3.00066(6) & 3.00055 & 3.00077 & \num{5.0e-03} & \num{2.5e-05}\\
\bottomrule
\end{tabular}
  \caption{Energy benchmark using Pade-Jastrow wave function, using $2^{22}$
    Monte Carlo samples and errors estimated by an automated blocking algorithm
    by~\textcite{Jonsson-2018}. See \cref{fig:QD-benchmark-pade-jastrow-training}
    for source code reference.}
  \label{tab:pade-jastrow-benchmark-energy}
\end{table}

\cref{fig:QD-benchmark-pade-jastrow-training} shows the optimization as function
of percentage of training completed. We can observe that the optimizations
quickly settles down to a set of optimal values, where it only oscillates
slightly back and forth. \cref{tab:pade-jastrow-benchmark-energy} shows
statistics for the energy obtained with the final state. Comparing to the
analytical result of $\SI{3}{\au}$ these results are in very good agreement. For
reference we have also given the results obtained without the Pade-Jastrow term,
i.e. the non-interacting ground state.

\section{Restricted Boltzmann Machine}

\begin{figure}[h]
  \centering
  \tikzstyle{neuron}=[draw,circle,minimum size=20pt,inner sep=0pt, fill=white]
\tikzstyle{stateTransition}=[thick]
\tikzstyle{learned}=[text=red]
\begin{tikzpicture}[scale=2]
  % \draw ;
  \draw[fill=black!30, rounded corners] (-0.2, -0.2) rectangle (3.2, 0.2) {};
  \draw[fill=black!30, rounded corners] (0.3, 0.8) rectangle (2.7, 1.2) {};

  \node (v1)[neuron] at (0, 0) {$x_1$};
  \node (v2)[neuron] at (1, 0) {$x_2$};
  \node (v3)[neuron] at (2, 0) {$x_3$};
  \node (v4)[neuron] at (3, 0) {$x_4$};
  \node[right=0.1cm of v4] (v) {$\vx \in \mathbb{R}^{P\cdot D}$};
  \node[learned,below=0.1cm of v1] (bv1) {$a_{1}$};
  \node[learned,below=0.1cm of v2] (bv2) {$a_{2}$};
  \node[learned,below=0.1cm of v3] (bv3) {$a_{3}$};
  \node[learned,below=0.1cm of v4] (bv4) {$a_{4}$};

  \node (h1)[neuron] at (0.5, 1) {$h_1$};
  \node (h2)[neuron] at (1.5, 1) {$h_2$};
  \node (h3)[neuron] at (2.5, 1) {$h_3$};
  \node[right=0.1cm of h3] (h) {$\vb{h} \in \{0, 1\}^3$};
  \node[learned,above=0.1cm of h1] (bh1) {$b_{1}$};
  \node[learned,above=0.1cm of h2] (bh2) {$b_{2}$};
  \node[learned,above=0.1cm of h3] (bh3) {$b_{3}$};

  \draw[learned,stateTransition] (v1) -- (h1) node [midway,above=-0.06cm,sloped] {$W_{1,1}$};
  \draw[stateTransition] (v1) -- (h2) node [midway,above=-0.06cm,sloped] {};
  \draw[stateTransition] (v1) -- (h3) node [midway,above=-0.06cm,sloped] {};

  \draw[stateTransition] (v2) -- (h1) node [midway,above=-0.06cm,sloped] {};
  \draw[stateTransition] (v2) -- (h2) node [midway,above=-0.06cm,sloped] {};
  \draw[stateTransition] (v2) -- (h3) node [midway,above=-0.06cm,sloped] {};

  \draw[stateTransition] (v3) -- (h1) node [midway,above=-0.06cm,sloped] {};
  \draw[stateTransition] (v3) -- (h2) node [midway,above=-0.06cm,sloped] {};
  \draw[stateTransition] (v3) -- (h3) node [midway,above=-0.06cm,sloped] {};

  \draw[stateTransition] (v4) -- (h1) node [midway,above=-0.06cm,sloped] {};
  \draw[stateTransition] (v4) -- (h2) node [midway,above=-0.06cm,sloped] {};
  \draw[learned,stateTransition] (v4) -- (h3) node [midway,above=-0.06cm,sloped] {$W_{4,3}$};
\end{tikzpicture}
  \caption{Example diagram of a Guassian-Binary Restricted Boltzmann Machine, showed with four visible nodes
    and three hidden nodes. The red values are the parameters, and consist of
    visible layer bias, $\vb a$, hidden layer bias, $\vb b$ and connection
    weights $\mat W$.}
  \label{fig:rbm-diagram-example}
\end{figure}


The first ML inspired model we have applied is a Restricted Boltzmann Machine (RBM).
This type of model has seen a significant rise in usage
since~\textcite{Carleo602} demonstrated the RBMs excellent capability to
represent the wave function for some selected Hamiltonians. However, all the
Hamiltonians for which they showed successful results were had a discrete
configuration space. The current system is continuous as the particles can have
any real valued coordinates, and so the type of RBM must change as well.

While several possible choices exist, we have used a Gaussian-Binary RBM. An
example diagram is given in~\cref{fig:rbm-diagram-example}. A full
introduction to RBMs, including the details needed for VMC can be found
in\cite{Flugsrud-2018}. For our purposes it suffices to say that the resulting
wave function looks as follows:

\begin{align}
  \label{eq:rbm-def}
  \psi_{RBM}(\vX) &=
        e^{-\sum_i^{M} \frac{\qty(X_i-a_i)^2}{2\sigma^2}}
        \prod_j^N \qty(1 + e^{b_j+\sum_i^M \frac{X_iW_{ij}}{\sigma^2}}),
\end{align}
where $M = P\cdot D$ is the number of degrees of freedom and $N$ is the number
of hidden nodes. Note also that $X_i$ in the above refers to the $i$'th degree
of freedom, counting through $\mat X$ in row major order. The parameters are
$\vb a, \vb b$ and $\mat W$, and $\sigma^2=1$ is held constant in this case.

If we set $\vb a = \vb 0$ we can recognize the first factor of~\cref{eq:rbm-def}
as the non-interacting ground state. That way we can consider the second factor
the Jastrow factor introduced by the RBM structure. It has a rather
unconventional form, as it is not a pure exponential.


\subsubsection{Optimizing}

We have run several variations of optimization on the RBM, all leading to rather
similar results. The following was produced with normally distributed random
initial values for the parameters, running $\num{60000}$ optimization steps with
$\num{2000}$ MC cycles each. We have also once again used importance sampling
and ADAM. A new addition (not strictly necessary for similar results) is the use
of mild $L2$ regularization, which serves to drive parameters that do not
contribute towards zero.

\cref{fig:QD-rbm-training} shows the ground state energy as a function of
training steps, along with the progression of the various variational
parameters. While we see a clear improvement in the initial stages, the RBM
fails to converge as accurately as the benchmark. \cref{tab:rbm-energy-results}
shows the precise results of the final model. While slightly different results
are possible with different training settings, we have never observed the RBM
achieve energies below $\SI{3.07}{\au}$, which is an error of about two orders
of magnitude larger than the benchmark.



% \begin{figure}[h]
%    \centering
%     \resizebox{\linewidth}{!}{%
%         % This file was created by matplotlib2tikz v0.7.4.
\begin{tikzpicture}

\definecolor{color0}{rgb}{0.12156862745098,0.466666666666667,0.705882352941177}
\definecolor{color1}{rgb}{1,0.498039215686275,0.0549019607843137}
\definecolor{color2}{rgb}{0.172549019607843,0.627450980392157,0.172549019607843}
\definecolor{color3}{rgb}{0.83921568627451,0.152941176470588,0.156862745098039}
\definecolor{color4}{rgb}{0.580392156862745,0.403921568627451,0.741176470588235}
\definecolor{color5}{rgb}{0.549019607843137,0.337254901960784,0.294117647058824}
\definecolor{color6}{rgb}{0.890196078431372,0.466666666666667,0.76078431372549}
\definecolor{color7}{rgb}{0.737254901960784,0.741176470588235,0.133333333333333}
\definecolor{color8}{rgb}{0.0901960784313725,0.745098039215686,0.811764705882353}

\begin{groupplot}[group style={group size=2 by 1}]
\nextgroupplot[
legend cell align={left},
legend style={draw=white!80.0!black},
tick align=outside,
tick pos=left,
x grid style={white!69.01960784313725!black},
xlabel={\% of training},
xmin=-5, xmax=105,
xtick style={color=black},
y grid style={white!69.01960784313725!black},
ylabel={Ground state energy (a.u.)},
ymin=2.98661068407909, ymax=3.28117563433902,
ytick style={color=black}
]
\addplot [semithick, color0]
table {%
0 3.26778631841812
1 3.14947938985174
2 3.08688912225823
3 3.08885451515261
4 3.09177410749784
5 3.09004525149916
6 3.0825541152983
7 3.08659830244641
8 3.08807346743545
9 3.08737072593149
10 3.08486888597368
11 3.08340761872365
12 3.08002103093302
13 3.08066509425232
14 3.08064724852931
15 3.0811325959243
16 3.08266140122991
17 3.07803865434052
18 3.07917375658203
19 3.07963901513708
20 3.07648725941841
21 3.08228553442701
22 3.07539106757631
23 3.07763127655871
24 3.07655428552753
25 3.07897182953364
26 3.08394594524133
27 3.08645322727406
28 3.082840787604
29 3.08068617245471
30 3.07836864155134
31 3.07921967769809
32 3.0790717484034
33 3.08201042139664
34 3.08488869616681
35 3.08604729702945
36 3.08348269965668
37 3.08521472731162
38 3.07849701860954
39 3.08451618888308
40 3.07831379169806
41 3.0817796559986
42 3.08292317194317
43 3.07993230566767
44 3.07640939000394
45 3.08087346092422
46 3.08058861137909
47 3.08001983078223
48 3.07761412644538
49 3.08610665192563
50 3.0813639637218
51 3.085521487051
52 3.0755401337017
53 3.08078022649304
54 3.07671814464576
55 3.08964303049499
56 3.0833480006293
57 3.07927864274738
58 3.07930757914846
59 3.07644874796548
60 3.07849958420558
61 3.08074328833594
62 3.07716301482793
63 3.08019863403575
64 3.08076955667335
65 3.08349780114468
66 3.0974741178163
67 3.07468187863078
68 3.07764140737187
69 3.0852065893175
70 3.08154362931524
71 3.0859197476721
72 3.08036312538626
73 3.08003121148469
74 3.07807609707624
75 3.08051735873207
76 3.07676654485076
77 3.077092807589
78 3.07723286080184
79 3.0774235331529
80 3.07924843845931
81 3.07832455119813
82 3.07745195139007
83 3.07494597671178
84 3.08148728802571
85 3.07865090379405
86 3.07976191585688
87 3.08699874854927
88 3.08083335491312
89 3.07882722252863
90 3.07886516337923
91 3.07881701675731
92 3.07454498423968
93 3.07734877499453
94 3.08619177869686
95 3.07443726356216
96 3.07844469746873
97 3.08239296752874
98 3.07912115499406
99 3.07938525071009
100 3.0787187826413
};
\addlegendentry{$\langle E_L\rangle$}
\addplot [semithick, black, opacity=0.5, dashed]
table {%
-5 3
105 3
};
\addlegendentry{Exact}

\nextgroupplot[
tick align=outside,
tick pos=left,
x grid style={white!69.01960784313725!black},
xlabel={\% of training},
xmin=-5, xmax=105,
xtick style={color=black},
y grid style={white!69.01960784313725!black},
ymin=-1.03556808440274, ymax=1.05857039605435,
ytick style={color=black}
]
\addplot [semithick, color0]
table {%
0 -0.117841969177496
1 -0.118284580859272
2 -0.147608196627473
3 -0.151476232340979
4 -0.161526852251141
5 -0.178177036893419
6 -0.184745040018205
7 -0.205690835570629
8 -0.231359339255245
9 -0.261957638762584
10 -0.291210806627067
11 -0.301131326969523
12 -0.345276125459561
13 -0.384083771771859
14 -0.426541570963376
15 -0.460732756268614
16 -0.521931907667269
17 -0.559987111617287
18 -0.588345868758849
19 -0.61298066432494
20 -0.654170151960175
21 -0.690641640079611
22 -0.686238687831283
23 -0.683465897548257
24 -0.695031030597753
25 -0.692580372744827
26 -0.706077028354332
27 -0.695129476076795
28 -0.700824232546135
29 -0.702201551573074
30 -0.701060363260462
31 -0.695954452438
32 -0.687008552536897
33 -0.703708909757854
34 -0.675566499950374
35 -0.690815051377925
36 -0.679702143904449
37 -0.694707347240377
38 -0.698010592748375
39 -0.698369986923307
40 -0.696915581834592
41 -0.674469260817328
42 -0.714023859148464
43 -0.708975374785558
44 -0.729003559361711
45 -0.691231814282918
46 -0.693192555380428
47 -0.704205376167331
48 -0.689699119861733
49 -0.714972001021041
50 -0.704259416870179
51 -0.694003624283448
52 -0.707638328510849
53 -0.698142704730003
54 -0.684634110943013
55 -0.705673541387552
56 -0.70165459652634
57 -0.712141716860302
58 -0.703220675026353
59 -0.696443139244194
60 -0.689258032975759
61 -0.706475314606333
62 -0.699144265460016
63 -0.689358733191036
64 -0.689816011697235
65 -0.700501424208051
66 -0.708612101944233
67 -0.694924513112424
68 -0.695563465278716
69 -0.686080509555422
70 -0.70408429278159
71 -0.709716623804077
72 -0.712428257596636
73 -0.703787042982799
74 -0.689308950583273
75 -0.691330331833291
76 -0.682303634294363
77 -0.697967874522511
78 -0.696416661192705
79 -0.696313872921052
80 -0.714946220503201
81 -0.705090159110704
82 -0.689035996798875
83 -0.699470311461095
84 -0.696675213706153
85 -0.686927489256884
86 -0.70551423437382
87 -0.700283695676133
88 -0.704363545298113
89 -0.67420772658133
90 -0.701153425631444
91 -0.702688101948318
92 -0.707625879200657
93 -0.701169936874863
94 -0.687166519895171
95 -0.705539786016749
96 -0.682868835354953
97 -0.703729777662518
98 -0.697633195405716
99 -0.705297109742872
100 -0.700929372490184
};
\addplot [semithick, color1]
table {%
0 -0.168517826694971
1 -0.116174681839336
2 -0.0237820443114784
3 -0.045509537555996
4 -0.0412411960005719
5 -0.0872885903411773
6 -0.106709285448678
7 -0.126759758439932
8 -0.148928359317894
9 -0.166584134981256
10 -0.168512585700758
11 -0.185253863692317
12 -0.179619950987432
13 -0.161906247273137
14 -0.164391107447321
15 -0.15570851047042
16 -0.141998136122209
17 -0.171690505265161
18 -0.177118655545456
19 -0.148698717936924
20 -0.170693321569304
21 -0.180295202934484
22 -0.175645852783201
23 -0.164622856533075
24 -0.182515761753992
25 -0.182042895013969
26 -0.178474934469043
27 -0.170422760971178
28 -0.168365442323457
29 -0.20541425216402
30 -0.176092545377143
31 -0.189588277974403
32 -0.186556648364522
33 -0.184450310761327
34 -0.193448632162503
35 -0.185826460518535
36 -0.180433077137332
37 -0.194779879746444
38 -0.176940685377482
39 -0.200109160438354
40 -0.185904039243833
41 -0.18609159048158
42 -0.1952880288162
43 -0.187120030418775
44 -0.184428202353413
45 -0.189501752870751
46 -0.175358827313395
47 -0.197793636458806
48 -0.196740663178943
49 -0.169963068431848
50 -0.193267988366865
51 -0.195139646173826
52 -0.184683854810395
53 -0.177255285552054
54 -0.199307678800736
55 -0.204875800953848
56 -0.19169146002929
57 -0.183554537215395
58 -0.204478776760284
59 -0.180106825643502
60 -0.180900075169887
61 -0.189459117613751
62 -0.205594335745631
63 -0.188013559718387
64 -0.199141270108549
65 -0.17994790736683
66 -0.203811415796159
67 -0.197864013888265
68 -0.197616899935467
69 -0.186473594374254
70 -0.200809573081615
71 -0.206724654667035
72 -0.209341474626318
73 -0.19975778493527
74 -0.192287680219868
75 -0.201015577272785
76 -0.194996816538647
77 -0.190931967861276
78 -0.181041097661161
79 -0.174470529288789
80 -0.200534364087405
81 -0.180116971053231
82 -0.174517753623464
83 -0.179247367938286
84 -0.197195547587956
85 -0.175300162764342
86 -0.201034810418223
87 -0.179814669328689
88 -0.184230338588902
89 -0.193777797324401
90 -0.175785531428736
91 -0.183712053164084
92 -0.168336728169393
93 -0.183289651000195
94 -0.180835853499765
95 -0.187224466381638
96 -0.185698440902889
97 -0.202071550282589
98 -0.19984109267011
99 -0.172071137777818
100 -0.186967150573917
};
\addplot [semithick, color2]
table {%
0 0.0578126847448953
1 0.0239691020502556
2 0.0622739143668992
3 0.0570222698023475
4 0.0766115907542125
5 0.0855841689657839
6 0.098824941580093
7 0.12404260098867
8 0.157952210276348
9 0.169676773156474
10 0.207968861719094
11 0.230189840426506
12 0.272328014975971
13 0.315552074549481
14 0.367348735513422
15 0.411174026554853
16 0.466970288079056
17 0.517286993678151
18 0.544717738654154
19 0.594325817092425
20 0.619897190632202
21 0.615301665469338
22 0.642263479369982
23 0.64685816029142
24 0.668979666327317
25 0.665464707695501
26 0.677092989994602
27 0.669739111359179
28 0.665372665595139
29 0.686206562071545
30 0.680784138975695
31 0.663049425837883
32 0.685685730521279
33 0.682999012945518
34 0.697306577707432
35 0.676219851886695
36 0.6940694604607
37 0.693411176357958
38 0.684868913939419
39 0.693262204414279
40 0.674702975246693
41 0.699503344408849
42 0.698529456918059
43 0.699509929604414
44 0.701204742477404
45 0.68749293877115
46 0.681013485011589
47 0.704964315856105
48 0.696725500881738
49 0.689733222743377
50 0.690633829368131
51 0.688879659097443
52 0.694521777154383
53 0.680883699072709
54 0.697819458740834
55 0.699421293105179
56 0.70012827394827
57 0.686586721374614
58 0.714291500144085
59 0.690147638269807
60 0.710268509671175
61 0.69364943022072
62 0.689880152479791
63 0.69427653051402
64 0.69171179927545
65 0.696162353132352
66 0.704802270963466
67 0.692995323445902
68 0.713897163949585
69 0.708002983068241
70 0.679513259300539
71 0.692966709888001
72 0.696895322421662
73 0.700481953698286
74 0.690581431273791
75 0.68536139221896
76 0.67802575188554
77 0.683093828649745
78 0.68758862651364
79 0.698301793251585
80 0.703021850798983
81 0.696429331260115
82 0.697988025385607
83 0.681965294949251
84 0.695703860951957
85 0.694252751572036
86 0.697020417853841
87 0.695613180216285
88 0.693806967167993
89 0.686742976325609
90 0.711861546827682
91 0.692136812380934
92 0.691350066816511
93 0.711624053866064
94 0.697124522625767
95 0.692834927783465
96 0.682736923440414
97 0.693754055894466
98 0.696698426316794
99 0.684833227965497
100 0.685116360081513
};
\addplot [semithick, color3]
table {%
0 0.183287074758229
1 0.132429481752042
2 0.0389640782807688
3 0.0657472578380956
4 0.0775976895420614
5 0.0862421974007416
6 0.142899213546159
7 0.161880437934315
8 0.173630092674005
9 0.201748690484137
10 0.195711198526497
11 0.192918318655036
12 0.180658409582405
13 0.175923834282281
14 0.19371460907686
15 0.183537293971219
16 0.177749852818179
17 0.180061551614179
18 0.177078272065152
19 0.181745805912424
20 0.164952620328332
21 0.175816500797101
22 0.171481607143277
23 0.174032037976757
24 0.186969669252365
25 0.182959411901719
26 0.187899813971458
27 0.191223494569405
28 0.185530437785418
29 0.19992167157473
30 0.188753704326813
31 0.188883420586678
32 0.205062510580702
33 0.17556783485388
34 0.218452013027873
35 0.198405965221406
36 0.194757077875679
37 0.208645904197481
38 0.187850879152027
39 0.194536709255283
40 0.19206095254628
41 0.170493529680027
42 0.204285057087975
43 0.196649398826768
44 0.195483748470771
45 0.195966988315664
46 0.16326969240153
47 0.194943976245223
48 0.189282911178688
49 0.223920866951644
50 0.185675556285813
51 0.206525635999128
52 0.192480185170143
53 0.176928706827285
54 0.195887311098028
55 0.187030604007873
56 0.184334721254913
57 0.20143965384844
58 0.194831948028397
59 0.220213152691589
60 0.192085332965006
61 0.187219815856679
62 0.192696270339148
63 0.183372063265722
64 0.195086686706335
65 0.187439941508585
66 0.200291479138498
67 0.192129476874698
68 0.195891251373809
69 0.186357463905647
70 0.191333810810809
71 0.195207560461176
72 0.208428608590187
73 0.198762035814046
74 0.178424729287923
75 0.171486782975363
76 0.192088065952442
77 0.184852921088306
78 0.196045956179313
79 0.181855183214294
80 0.18586940988296
81 0.185040377825289
82 0.187500043649283
83 0.193373988336058
84 0.168744836639204
85 0.199614890095825
86 0.172481725649594
87 0.166273150543282
88 0.179830347437683
89 0.198731472953572
90 0.195616771862649
91 0.188804362480916
92 0.190846083876476
93 0.183124611766118
94 0.194625111549432
95 0.168130154502491
96 0.183853267131626
97 0.183502983448932
98 0.179683757825927
99 0.185130876358602
100 0.180034979038336
};
\addplot [semithick, color4]
table {%
0 -0.137779610369542
1 -0.136937979892395
2 -0.0883592401185548
3 -0.0759296966030259
4 -0.0520952958777989
5 -0.0307572076762653
6 -0.0759098785997443
7 -0.0824171842469352
8 -0.0155793500732775
9 -0.0444361235560702
10 -0.0415661812123278
11 -0.0539319774248856
12 -0.0443577980583036
13 -0.0408636053242552
14 -0.0363023492039638
15 -0.0283237187440424
16 -0.0551033435805871
17 -0.0580216911782531
18 -0.0662746524076453
19 -0.0358261670773747
20 -0.056466627836234
21 -0.027626949040093
22 -0.0340606595746497
23 -0.00470877211849913
24 -0.0522075868117601
25 -0.00969091780504367
26 -0.0179670349525436
27 0.0178047762034177
28 0.00221114492838966
29 -0.00182677270046398
30 -0.00258202818677059
31 -0.00698757875348517
32 -0.0470266457302778
33 -0.0154395827988023
34 -0.0649750606099243
35 -0.0433032734994719
36 -0.0355542832793026
37 -0.0290234386901202
38 -0.0339227902530721
39 -0.0246360090291442
40 -0.00237390721626307
41 -0.00208697112403829
42 -0.0368384156375818
43 -0.0094282254607214
44 -0.0478995651191059
45 -0.0114003769919481
46 0.0256708807928039
47 -0.0303861655852825
48 0.0100238448867629
49 -0.0291563639278864
50 -0.0389513202755132
51 -0.0318823925294402
52 -0.0280949188427158
53 -0.00629452836389987
54 -0.00688472858890938
55 0.00988212723348274
56 0.00530323375829856
57 -0.0133038248122804
58 -0.0213183632799106
59 -0.00116352769881812
60 -0.000142107558771265
61 0.00326342918826165
62 -0.0127645112409538
63 -0.00263068989998797
64 -0.0260467020749329
65 -0.0122859943274085
66 -0.0232273498230273
67 -0.00398018816928315
68 -0.0235829687401191
69 0.0153238724460127
70 -0.0247437816172799
71 -0.0254275867185743
72 -0.0348281956711
73 -0.0312713204826211
74 -0.00651885620782185
75 -0.0264032175610295
76 -0.0265624955381769
77 -0.0190003766864374
78 -0.0016543028749428
79 -0.0334193254670314
80 -0.0457731292803179
81 -0.0162721930347208
82 -0.00697863918601924
83 0.0374349667463795
84 -0.00535457039306125
85 0.0115695225478509
86 -0.034423773615372
87 0.0092139748117145
88 0.00652262783093436
89 0.00844333057161675
90 -0.0528632334485723
91 -0.000694414145357467
92 0.00513653996283588
93 -0.0280267427662427
94 -0.0384508736062291
95 -0.021214529408253
96 -0.0141942491109538
97 -0.0611973591139666
98 -0.0469820600735964
99 -0.00343150076332478
100 -0.0100598469829275
};
\addplot [semithick, color5]
table {%
0 0.035092500313306
1 -0.0861822678049885
2 -0.127367559245924
3 -0.0874432623784901
4 -0.0817434508706693
5 -0.0677911905187272
6 -0.0398810650801974
7 -0.0329043021735603
8 -0.0353803327205786
9 -0.0533234745522923
10 -0.090276221510884
11 -0.102310274460065
12 -0.111574468577508
13 -0.134491025470206
14 -0.132825894110224
15 -0.131113541263183
16 -0.137361328468543
17 -0.126419595530195
18 -0.136936252486358
19 -0.13233361652405
20 -0.158313455095911
21 -0.183002693496246
22 -0.186348569980816
23 -0.182159487198703
24 -0.333772717673867
25 -0.248020487380919
26 -0.214044951930013
27 -0.240915024687131
28 -0.224510177823057
29 -0.187896083112768
30 -0.209737340484752
31 -0.239848962320034
32 -0.221481744414614
33 -0.199415044734208
34 -0.193155609107824
35 -0.241408146883695
36 -0.234367035994861
37 -0.261107638692642
38 -0.229054608895576
39 -0.256194031880803
40 -0.241640730291495
41 -0.260972439938678
42 -0.233617605175208
43 -0.280258532639826
44 -0.255621331081189
45 -0.26723505544152
46 -0.242793935706818
47 -0.257278401483627
48 -0.20321653037851
49 -0.296042381621783
50 -0.234789191028638
51 -0.26878041394595
52 -0.317425431204237
53 -0.226655613045935
54 -0.246049091123516
55 -0.167707909475823
56 -0.166000201349949
57 -0.224422461787242
58 -0.262277114338316
59 -0.275686160121517
60 -0.272437512637225
61 -0.27801710012176
62 -0.19010269660309
63 -0.25378171451453
64 -0.250490341708053
65 -0.338313182760493
66 -0.247381257872765
67 -0.225399323300402
68 -0.228418423373035
69 -0.278898001989736
70 -0.297403857659968
71 -0.217442623976893
72 -0.186559627010419
73 -0.381647814133918
74 -0.218130555788455
75 -0.27657113953966
76 -0.175979632708966
77 -0.20595318553399
78 -0.244679050154886
79 -0.199933883338751
80 -0.193636105616554
81 -0.253127018589654
82 -0.189180117914464
83 -0.284267265009544
84 -0.240835642469927
85 -0.231376727625655
86 -0.265365733051995
87 -0.232930566479254
88 -0.209480340257955
89 -0.183801115357132
90 -0.197981746911258
91 -0.210579277760183
92 -0.266103009393965
93 -0.299393719404831
94 -0.270869937841552
95 -0.245823284080621
96 -0.241555787415371
97 -0.288541603303101
98 -0.20221207262901
99 -0.263991224438291
100 -0.269974079519619
};
\addplot [semithick, color6]
table {%
0 -0.144743482433268
1 -0.134061025577377
2 -0.0624524383444256
3 -0.0681789751772014
4 -0.0532538392091842
5 -0.0548186511245464
6 -0.083748389257064
7 -0.0330254199856477
8 -0.0192767990627175
9 -0.0334528557389933
10 0.0115983678495587
11 0.0316925782446146
12 0.0447459203568569
13 0.0756848178763574
14 0.098959916117016
15 0.10497911731242
16 0.127978706161942
17 0.127772245430811
18 0.103781647685905
19 0.0841660213868786
20 0.0363624565440729
21 -0.0183061294592857
22 -0.0593735030054122
23 -0.0910754542021116
24 -0.238182000376647
25 -0.208877785515179
26 -0.194263930373288
27 -0.21649839295981
28 -0.210430664533909
29 -0.188652298204955
30 -0.20126210368189
31 -0.234419049721137
32 -0.215163998009109
33 -0.204539616651082
34 -0.182026908678248
35 -0.236154399002207
36 -0.224507125359404
37 -0.25471894988242
38 -0.229568378223817
39 -0.25637613822981
40 -0.241548486750939
41 -0.259006922749968
42 -0.234332962121301
43 -0.277875054385455
44 -0.256708262715272
45 -0.26867289643004
46 -0.243454718096918
47 -0.257109138950448
48 -0.203644412066994
49 -0.293780891357521
50 -0.23508798487324
51 -0.268292758576046
52 -0.317603241865034
53 -0.226820815825593
54 -0.24595076188124
55 -0.168002936623127
56 -0.166135509324588
57 -0.2241861914102
58 -0.262196732080478
59 -0.275403733655406
60 -0.272312896564855
61 -0.27804726837267
62 -0.190158413882081
63 -0.25375477719795
64 -0.25048875168738
65 -0.338289084309062
66 -0.247381714603421
67 -0.225396895294042
68 -0.228401027980755
69 -0.278860079019857
70 -0.297423521240624
71 -0.21745501861486
72 -0.186560279147583
73 -0.381634564352989
74 -0.218154132414842
75 -0.27658582498708
76 -0.175984654043789
77 -0.205965263099859
78 -0.244676445215255
79 -0.199929117162813
80 -0.193647960884266
81 -0.253127402765016
82 -0.189178016783298
83 -0.284257554628864
84 -0.240842399558694
85 -0.231370080538972
86 -0.26537095880168
87 -0.232933277972366
88 -0.209481569535001
89 -0.183799012285928
90 -0.197980178561512
91 -0.210578678178436
92 -0.2661008562555
93 -0.299393402294522
94 -0.270868944303222
95 -0.245824464826369
96 -0.241556030773985
97 -0.288542602787214
98 -0.20221305261247
99 -0.26399085282688
100 -0.269974472831171
};
\addplot [semithick, white!49.80392156862745!black]
table {%
0 -0.0151524271686803
1 -0.0913183015721587
2 -0.129856608124111
3 -0.094513391348125
4 -0.0814417795742177
5 -0.0798100558368352
6 -0.0207877738007846
7 -0.0614710584400028
8 -0.0494776666981612
9 -0.0405190492766855
10 -0.0640515031014897
11 -0.0418542694148258
12 -0.0324149145043087
13 -0.0608249625773501
14 -0.0563851811238518
15 -0.0421377574057834
16 -0.0748456286785623
17 -0.0761960775966097
18 -0.0825163439262087
19 -0.0825349927451225
20 -0.0708469227157535
21 -0.069209734212321
22 -0.0404906719870282
23 -0.0224742060570746
24 -0.0358067330866515
25 -0.0282608608993036
26 -0.0281801595944008
27 -0.0210811043047623
28 -0.02568620342767
29 0.00719083918959043
30 -0.0262925541095772
31 -0.0182378692159596
32 -0.0167911334550508
33 -0.0185180804419946
34 -0.0123037292229249
35 -0.0205161496917184
36 -0.00124669763006091
37 -0.0191821070880497
38 -0.0357413416294931
39 -0.0432993717364893
40 -0.0488351836555899
41 -0.00771328557611957
42 -0.00730029439162058
43 0.0143939606112296
44 -0.00349587120197966
45 0.0458464975662798
46 -0.00405767532845159
47 -0.0400172798834284
48 -0.0502253921782351
49 -0.0475557330276617
50 -0.0391037794909719
51 -0.034299568300274
52 -0.0398705577822238
53 -0.040362880001581
54 -0.0157456088107293
55 -0.0512795324207387
56 -0.0525436117977158
57 -0.0147237930536626
58 -0.033661132756271
59 -0.0174786131764528
60 -0.0301333122407204
61 -0.0223624406192437
62 -0.0234556886979485
63 -0.0276418579090159
64 -0.00334814705175983
65 -0.0391119305709429
66 -0.0186081407873672
67 -0.0392296949098633
68 -0.0472601072470643
69 -0.0505223848724599
70 -0.030787989733405
71 -0.0226784850273241
72 -0.0222959882703397
73 -0.00576706370624396
74 -0.0255646041380676
75 -0.0212629945410975
76 -0.00339079428875706
77 -0.0207011727257665
78 -0.0148346241088344
79 -0.0349322250926083
80 -0.046149790962858
81 -0.0288791024465654
82 -0.0310033589088809
83 -0.00540455207237805
84 0.00434757825070736
85 -0.0391035678548132
86 -0.0571549329226187
87 -0.030617286302665
88 -0.0370086522491138
89 0.0100021024359211
90 -0.0215556143500916
91 -0.000846118544716431
92 -0.0124659809726186
93 -0.022423652439006
94 -0.0422869068261843
95 -0.0443451795715016
96 -0.00496974330239706
97 -0.0151316062536292
98 -0.0357564793598738
99 -0.0248755861628915
100 -0.0170723701890036
};
\addplot [semithick, color7]
table {%
0 0.185740813824724
1 0.564400702060892
2 0.430310241373172
3 0.428117645181848
4 0.438587903219415
5 0.435875734664948
6 0.469018381906866
7 0.465244558600847
8 0.474669606916768
9 0.497878121318511
10 0.489189879457369
11 0.481191026407585
12 0.491685914324221
13 0.488078909144776
14 0.536714207402653
15 0.542254490394387
16 0.531540517712119
17 0.504225646427518
18 0.513711201943898
19 0.526824265206809
20 0.517419939237025
21 0.53900177820525
22 0.52268899899133
23 0.515845763196493
24 0.50464382105214
25 0.543672965887466
26 0.522965894666593
27 0.534148312353307
28 0.548355570378578
29 0.540135073123262
30 0.510093954667188
31 0.513944745940331
32 0.529567350989963
33 0.500872803286468
34 0.534315445888809
35 0.523953481500412
36 0.529418080685294
37 0.519232594444793
38 0.503867862033345
39 0.520242329939718
40 0.503152676136154
41 0.541175219672546
42 0.528651060143593
43 0.509392576261098
44 0.537660479636029
45 0.515712594445901
46 0.511512778367004
47 0.501740072620905
48 0.513867497010957
49 0.500456803897535
50 0.515300827539893
51 0.492120130819322
52 0.497955498558031
53 0.519310036084187
54 0.518943722680145
55 0.515656505687258
56 0.511396780405527
57 0.516546250967337
58 0.521396810297285
59 0.520308585548292
60 0.516098437422695
61 0.505658073275023
62 0.503866374166173
63 0.529444155652338
64 0.508675030170382
65 0.490082825430173
66 0.507607682720173
67 0.502775154645144
68 0.507398593950118
69 0.506988845612875
70 0.520679540209823
71 0.522484653824316
72 0.518673610308731
73 0.525084311009008
74 0.501635952824381
75 0.517308135666404
76 0.51788874964763
77 0.518435663898113
78 0.534758626915488
79 0.516302772846408
80 0.53752440103309
81 0.513931210459752
82 0.524659342557633
83 0.51108529273048
84 0.511335054721793
85 0.54160604151172
86 0.522644085167516
87 0.547571737272739
88 0.540710202947581
89 0.535843884776325
90 0.526085596817978
91 0.517610867641372
92 0.506837434116663
93 0.523400424636136
94 0.542384889365546
95 0.529725847968781
96 0.528757290055282
97 0.509835821422129
98 0.538096588069685
99 0.528950118300225
100 0.500594568043341
};
\addplot [semithick, color8]
table {%
0 -0.127210769859985
1 -0.309143690842585
2 -0.188826680942468
3 -0.158572728032208
4 -0.140293417387307
5 -0.108414078467205
6 -0.0801318756084798
7 -0.0567380121892827
8 -0.047824128027717
9 -0.0539157313604611
10 -0.054481490678503
11 -0.0380655763810121
12 -0.0543385829093575
13 -0.0622424873488257
14 -0.0692721360269495
15 -0.0591751147235655
16 -0.0735407127228184
17 -0.0729947374824756
18 -0.0581314812009485
19 -0.0456712197959816
20 -0.0498459474679266
21 -0.060854904329488
22 -0.0295471532432703
23 -0.00814874941995223
24 -0.0233355797359141
25 -0.0188323703804001
26 -0.0231643292462584
27 -0.00863335945049996
28 -0.00883653331384325
29 -0.00800593920548077
30 -0.00590329127799977
31 -0.00519576971252494
32 0.00348387075806275
33 -0.0117733429064856
34 0.0142807961376137
35 -0.00660510436865664
36 0.00380009820059887
37 -0.009965683026366
38 -0.0111539177833635
39 -0.0164972866930593
40 -0.0122474389589944
41 0.0102311980762148
42 -0.0293032618328811
43 -0.0215051133404883
44 -0.0398270003101616
45 -0.000191586499098079
46 -0.00429697863792422
47 -0.0130404299433696
48 -0.000264251694213936
49 -0.0259074605802648
50 -0.0156617984326376
51 -0.006646851972628
52 -0.019634067421405
53 -0.0148114951913364
54 0.00125225703311752
55 -0.0205354510387476
56 -0.016847810845111
57 -0.0277727601310845
58 -0.0175763858361863
59 -0.00903657486146453
60 -0.000851968280667919
61 -0.017085798426101
62 -0.0116572297014212
63 -0.00432047737728083
64 -0.0108480948647036
65 -0.0157315902168654
66 -0.0225148455115336
67 -0.009228344362916
68 -0.0123376872690355
69 -0.0064434591952364
70 -0.0218464421051398
71 -0.0235295636679264
72 -0.0252897292464488
73 -0.0135355073867645
74 -0.00103290726253009
75 -0.00589297483913712
76 0.00160269483220595
77 -0.0158498793743093
78 -0.0121988610128822
79 -0.0105638943568302
80 -0.0335250825957725
81 -0.0226292970969769
82 -0.00361318899144701
83 -0.0135086532788023
84 -0.00898121225808707
85 0.00261872557247295
86 -0.0155579236356483
87 -0.0108620021018556
88 -0.0166523091378595
89 0.00906685224803018
90 -0.0198552549593788
91 -0.0177595452561443
92 -0.0215167491317527
93 -0.0103443128713919
94 -0.00153438099054566
95 -0.019482683591434
96 0.0049994971437157
97 -0.018027928959318
98 -0.0159147229871682
99 -0.0226839987749557
100 -0.0169886715777548
};
\addplot [semithick, color0]
table {%
0 -0.020279840296916
1 -0.465008797275946
2 -0.562795180415626
3 -0.559117975224008
4 -0.5843409654271
5 -0.581550341278437
6 -0.578623512955776
7 -0.577146525639188
8 -0.579826661860961
9 -0.567841793613865
10 -0.571547844673873
11 -0.514463595255402
12 -0.47500228755139
13 -0.422908379944575
14 -0.363429088031257
15 -0.282393408169998
16 -0.228823596601338
17 -0.176746753103247
18 -0.119488317535597
19 -0.0803314963541553
20 -0.0655633849222915
21 -0.0631002859732162
22 -0.0270530573809602
23 -0.00265011054300263
24 -0.0119762417098525
25 -0.00787145990419015
26 -0.0107666561695155
27 0.00355906478026639
28 0.00415376336340823
29 0.00325081422590864
30 0.00601859856214954
31 0.00641564639006571
32 0.0128452720434266
33 -0.00393157281217167
34 0.0211019756835126
35 -7.67619142407382e-05
36 0.00940992091153806
37 -0.00565149194971095
38 -0.0083087149178576
39 -0.0142736639619324
40 -0.0101091498947752
41 0.0121248777819785
42 -0.0278664870668419
43 -0.0201193404306347
44 -0.0391541006155847
45 0.000630560996234969
46 -0.0034324039297588
47 -0.0123845280403603
48 0.000314112922149989
49 -0.0253965964595373
50 -0.015091358195161
51 -0.00614178364264215
52 -0.0194131221016013
53 -0.0145886181770635
54 0.00144265278117427
55 -0.0203555784762591
56 -0.0166773232009295
57 -0.0276536893499219
58 -0.0174666146795075
59 -0.00893696161228713
60 -0.000760919689991239
61 -0.0169960401968984
62 -0.0115675792672314
63 -0.00424196897442323
64 -0.0107819214646101
65 -0.01568995209281
66 -0.0224696705754084
67 -0.00919162864766452
68 -0.0123028456500246
69 -0.00641273990613613
70 -0.0218214536399988
71 -0.0235128783095538
72 -0.0252709478199874
73 -0.0135245886947318
74 -0.00102052319654432
75 -0.00588111681737646
76 0.00161251297727835
77 -0.0158400273814957
78 -0.0121926506513113
79 -0.0105588663941946
80 -0.0335207181066437
81 -0.0226260609871968
82 -0.0036094931791747
83 -0.0135061796639012
84 -0.00897921445227693
85 0.00262078444118238
86 -0.0155558831042994
87 -0.0108599950155236
88 -0.0166505232469969
89 0.00906815205296643
90 -0.0198536458234629
91 -0.0177581409005708
92 -0.0215157217923554
93 -0.0103439811537317
94 -0.00153407750684499
95 -0.0194823337603811
96 0.0049999107172834
97 -0.0180275989575148
98 -0.0159145472362228
99 -0.0226837792332079
100 -0.0169885564108761
};
\addplot [semithick, color1]
table {%
0 0.089613060480085
1 0.599916449698758
2 0.653101420105265
3 0.641011159104924
4 0.653996479980398
5 0.631802814370314
6 0.627734442200532
7 0.623436031121271
8 0.627274539205052
9 0.622714808987998
10 0.676072163991768
11 0.71187724014807
12 0.722664345700417
13 0.759443019645218
14 0.784875099929295
15 0.824963899376292
16 0.824431435360996
17 0.863931008779328
18 0.858757460821442
19 0.894505324965625
20 0.88371741888048
21 0.86484373730374
22 0.914551623468448
23 0.881962569657938
24 0.874253941190495
25 0.912267666682708
26 0.89477296302046
27 0.922021058737191
28 0.900563601878444
29 0.908548842403508
30 0.903001296362328
31 0.910609481050287
32 0.933509760801212
33 0.909326152159307
34 0.949051308403958
35 0.916080505611201
36 0.924718926321261
37 0.92067333729323
38 0.892635146688145
39 0.90847995822652
40 0.924291669858544
41 0.937963236571184
42 0.896804815123866
43 0.907657578599111
44 0.925589433985725
45 0.91319661578445
46 0.930800281811188
47 0.930692331380574
48 0.921178428093383
49 0.927181752630266
50 0.905584916886644
51 0.896196223952069
52 0.897100253804536
53 0.897189454656316
54 0.927662264854524
55 0.910071947270229
56 0.905163246233428
57 0.913742504547583
58 0.90985039343567
59 0.90879911457184
60 0.896658803917826
61 0.886546250741537
62 0.909883865150861
63 0.901732924837414
64 0.915317373638976
65 0.911016037804162
66 0.93330987725571
67 0.919937532934732
68 0.910684245341779
69 0.933024483041163
70 0.899159498308535
71 0.904314796640267
72 0.911360557431201
73 0.928482226681151
74 0.886640294366999
75 0.90589967070095
76 0.875225548915673
77 0.909464024954407
78 0.917912997370706
79 0.899171760758954
80 0.899892433270136
81 0.889566799354187
82 0.90138867882682
83 0.897722359023062
84 0.905559250057484
85 0.908556039876809
86 0.896235633855294
87 0.925018515028476
88 0.894950089899176
89 0.912545318801676
90 0.913724604894645
91 0.897870086971606
92 0.893740317648763
93 0.896997605907204
94 0.909360567750601
95 0.921724991595671
96 0.918565111565036
97 0.887718350568724
98 0.917827313012725
99 0.909787225151165
100 0.879738633903029
};
\addplot [semithick, color2]
table {%
0 -0.0279542097326349
1 0.134209630403903
2 0.62475017958877
3 0.63882229447482
4 0.730277758497982
5 0.718095469950545
6 0.814563665848605
7 0.833260300993434
8 0.873362860718765
9 0.883809298851863
10 0.896499521955441
11 0.883024443877646
12 0.872557439123461
13 0.90050895237022
14 0.893882205622181
15 0.900217529010745
16 0.914494638099379
17 0.891546594397563
18 0.868961030906912
19 0.897969428793512
20 0.869224928965347
21 0.908158501516232
22 0.917535739253525
23 0.897586060917683
24 0.893948680713452
25 0.919210702115024
26 0.896500494435855
27 0.930665719210847
28 0.920105806764833
29 0.89777252930814
30 0.907110415884091
31 0.890393682432456
32 0.907309356979427
33 0.902575346816189
34 0.943088847568461
35 0.932637640451213
36 0.916157752926834
37 0.909751689383579
38 0.905100132795156
39 0.883286242217927
40 0.927701604337379
41 0.904512240349471
42 0.88587031220653
43 0.907680682212569
44 0.914980898777978
45 0.912488676949559
46 0.886022119334003
47 0.894588088022476
48 0.90590556655255
49 0.963382283306302
50 0.905365360089846
51 0.881369497137348
52 0.904085228259299
53 0.921547080902907
54 0.889658611494479
55 0.902484140379083
56 0.910960583571232
57 0.920707161829616
58 0.885503978827601
59 0.905573593230947
60 0.918644081971972
61 0.894098147843676
62 0.892496713646384
63 0.9281073651124
64 0.91365801951057
65 0.909334715598912
66 0.931101832211137
67 0.930462728958704
68 0.916984376450327
69 0.939977025035324
70 0.909354775270575
71 0.908990000304543
72 0.930530334074
73 0.906281081841937
74 0.896915893909362
75 0.895271608206634
76 0.874579080548729
77 0.909779453191321
78 0.918001228634447
79 0.899066848382685
80 0.874886396663427
81 0.913729277139789
82 0.891519087403049
83 0.904552215164188
84 0.87749140760206
85 0.909827062562527
86 0.904837270430632
87 0.914358063743227
88 0.904840286320125
89 0.90905244322526
90 0.897661790575473
91 0.887496443725669
92 0.920658459551801
93 0.899529061394833
94 0.902756006139109
95 0.913927136838617
96 0.887405244936184
97 0.893483099973008
98 0.908929503569744
99 0.905407728623891
100 0.897450134177544
};
\addplot [semithick, color3]
table {%
0 -0.0822081288091147
1 -0.346651911423352
2 -0.613973929273692
3 -0.583697372936934
4 -0.54580067978958
5 -0.462202641785364
6 -0.39310846656047
7 -0.282091061054252
8 -0.200618079115212
9 -0.131903728900105
10 -0.0846827424966048
11 -0.0752599177127692
12 -0.0565706599683982
13 -0.0338994620411036
14 -0.0355401418884626
15 -0.0271387208405953
16 -0.0173825473395991
17 -0.0396719472974698
18 -0.043491553116733
19 -0.0122127522917243
20 -0.0351012158848515
21 -0.0443808530531666
22 -0.0364420530904084
23 -0.0204151430559643
24 -0.027643958466546
25 -0.0252121126962793
26 -0.0154278898533508
27 -0.00442217780854604
28 0.0014940977257117
29 -0.0343473289800796
30 -0.000586275611184962
31 -0.0139843003714946
32 -0.00703544294231129
33 -0.00651120837518475
34 -0.0126665368862524
35 -0.00159858881491584
36 0.00499414097312708
37 -0.00975931829823363
38 0.00613925886894319
39 -0.0189310800805591
40 -0.00362977497129052
41 -0.00562518998832603
42 -0.0132811610327428
43 -0.00419845832860246
44 -0.000620929430660461
45 -0.00877318704970347
46 0.00796219304097522
47 -0.0163822938579278
48 -0.0149607014446803
49 0.0143036380657068
50 -0.00579365416537864
51 -0.00651950014918912
52 0.00528448908337767
53 0.00858413932192571
54 -0.012154374765806
55 -0.0156366620484745
56 -0.00219513870928968
57 0.00558631009293842
58 -0.0147113152890607
59 0.0137187408914387
60 0.0100520927401675
61 0.00184796203895028
62 -0.0173881173140804
63 0.00101428274714452
64 -0.0086092045719831
65 0.0132649665369319
66 -0.00929570395723067
67 -0.00383341562009472
68 -0.00260610137423952
69 0.0124702758883183
70 0.000941608142556405
71 -0.0066516523732524
72 -0.0120202085935801
73 -0.000618976933828694
74 0.00310160936980089
75 -0.00698026772276314
76 -0.00364121152095907
77 -0.00373651961508492
78 0.00702765329634606
79 0.012112332955304
80 -0.0161034221352298
81 0.00742446087549351
82 0.0104535086939507
83 0.00363690248520978
84 -0.0147797778646246
85 0.00482704678548574
86 -0.0201034552833827
87 -0.00130004364921246
88 -0.00405056000518497
89 -0.0127794533251937
90 0.00636462532068044
91 -0.000420571899323866
92 0.0138441174486399
93 -0.00116137882908327
94 -0.00360834925019078
95 -0.00952002401411433
96 -0.00797248009618516
97 -0.0229859719015036
98 -0.0205690327334577
99 0.012492225284552
100 -0.000639768101075727
};
\addplot [semithick, color4]
table {%
0 0.0523299263871211
1 0.0128402935429864
2 0.32342157371701
3 0.289422729644441
4 0.314175580038666
5 0.227568102911812
6 0.216440799713615
7 0.173543988576925
8 0.13254503074183
9 0.0860644988888948
10 0.078055776203139
11 0.063503910505414
12 0.0389137473355082
13 0.0368937634284292
14 0.0102509695782879
15 0.016405948176071
16 0.0203988695158986
17 -0.00298388231655481
18 -0.010351416394471
19 0.0227874487940058
20 0.000809415931047931
21 -0.0105614828183837
22 -0.00429303632583092
23 0.00973207766091274
24 -0.00222758123530619
25 -0.000824266412340976
26 0.00766992838731309
27 0.0176239673401782
28 0.0221739330151497
29 -0.0146833560614975
30 0.0172923213839767
31 0.00416100934549832
32 0.00910453396438521
33 0.00629073369541717
34 -0.000942629044429139
35 0.00797679722128
36 0.0134417781174868
37 -0.00296664170340147
38 0.0117462194130336
39 -0.0145666110262433
40 0.000259996763272621
41 -0.00187859052680383
42 -0.00975176001702582
43 -0.000887891669975424
44 0.00198943794752428
45 -0.00656648626039644
46 0.00969667475232153
47 -0.0148630849090805
48 -0.0134953146989408
49 0.0155060132961252
50 -0.00476048330653074
51 -0.0055858344297903
52 0.00597891422360206
53 0.00919002482708495
54 -0.0116483038356547
55 -0.0151896007480186
56 -0.00181278712454487
57 0.0059092951660885
58 -0.0144453034534376
59 0.0139381862679738
60 0.0102309968887791
61 0.00201753519594805
62 -0.0172439464687825
63 0.00114170676873896
64 -0.00850119270132762
65 0.0133424816518456
66 -0.00922631476540917
67 -0.00377504844152722
68 -0.0025561089413502
69 0.0125189274961834
70 0.00099197637265555
71 -0.00660308963753504
72 -0.011972915130545
73 -0.000583864279624046
74 0.00313528944349345
75 -0.00695249658777776
76 -0.00361707897470131
77 -0.00371596475207699
78 0.0070454090949515
79 0.0121284233593077
80 -0.0160887904682407
81 0.00743818979196209
82 0.0104656573655064
83 0.00364809886455187
84 -0.0147695549372568
85 0.00483604966202321
86 -0.0200956823445592
87 -0.00129350174052566
88 -0.00404468135150173
89 -0.0127742222997899
90 0.00636896864344917
91 -0.000416565070560647
92 0.013847320029212
93 -0.00115853243026959
94 -0.00360599112997553
95 -0.00951792195933171
96 -0.00797064135254995
97 -0.0229843911005374
98 -0.0205676012091926
99 0.0124934267957872
100 -0.000638737079942881
};
\addplot [semithick, color5]
table {%
0 0.0239079946777208
1 -0.0618757316859885
2 -0.248625499978308
3 -0.274901869237352
4 -0.318681457142737
5 -0.346591527123728
6 -0.418754127574017
7 -0.453147463226877
8 -0.486950614352919
9 -0.500278716959613
10 -0.509859902210839
11 -0.546376891904615
12 -0.534164148660382
13 -0.521516909603469
14 -0.529037908725924
15 -0.527925871256656
16 -0.502641318040648
17 -0.540624657050833
18 -0.537351049668862
19 -0.532739342705799
20 -0.530843196647544
21 -0.543248585464475
22 -0.541267263780996
23 -0.524219647704841
24 -0.51799808641966
25 -0.532562656024279
26 -0.506663731805684
27 -0.506618116646313
28 -0.496636753465889
29 -0.544423219404931
30 -0.504495681192598
31 -0.492091784273914
32 -0.534265030404516
33 -0.499800446765691
34 -0.539266855870389
35 -0.498181860170824
36 -0.499038158722337
37 -0.522793555440423
38 -0.519348136118122
39 -0.516234803017046
40 -0.547747565740822
41 -0.540139481986765
42 -0.520503475624764
43 -0.527858574731362
44 -0.520166086298114
45 -0.524606175422114
46 -0.508889807256558
47 -0.505872955782844
48 -0.519375145108895
49 -0.535718065050662
50 -0.505450966677439
51 -0.503974898615646
52 -0.531094262645444
53 -0.500102989205846
54 -0.533881343635237
55 -0.519373653912878
56 -0.519189809902375
57 -0.512356243340645
58 -0.503441597618907
59 -0.516433914276847
60 -0.498539168505479
61 -0.529596801537499
62 -0.535065225885555
63 -0.495240312377855
64 -0.508888196660243
65 -0.474981268547323
66 -0.489335137148831
67 -0.511128914367795
68 -0.504086080420057
69 -0.489578929663015
70 -0.50142917783661
71 -0.535072699797056
72 -0.513056595045866
73 -0.527682712217423
74 -0.520150853543834
75 -0.528124660936381
76 -0.507152784898158
77 -0.522808030948871
78 -0.523139916485669
79 -0.496931110771262
80 -0.497419899853149
81 -0.517139799189877
82 -0.515132869765839
83 -0.524195764462053
84 -0.537874379923099
85 -0.519184062662876
86 -0.561539751240517
87 -0.536355504161136
88 -0.516325352402265
89 -0.54169997168563
90 -0.500384978502524
91 -0.524253189913617
92 -0.5129961277287
93 -0.523078650686993
94 -0.522961554775772
95 -0.523968858844558
96 -0.527851321649046
97 -0.525699424607787
98 -0.538945098479988
99 -0.501375479215407
100 -0.527323569589183
};
\addplot [semithick, color6]
table {%
0 0.131665395014097
1 -0.359038848447115
2 -0.362095994244238
3 -0.399549911366604
4 -0.405676686514564
5 -0.404497103833507
6 -0.426075610930432
7 -0.444602875209756
8 -0.446036810250043
9 -0.443873463180679
10 -0.469315752665445
11 -0.467910125352682
12 -0.495363417457397
13 -0.494145161166332
14 -0.510580156856904
15 -0.493557745610018
16 -0.514831503087566
17 -0.505470110222464
18 -0.518526655654396
19 -0.508231433784856
20 -0.510808195910495
21 -0.538508040537957
22 -0.518944929970836
23 -0.503974838385426
24 -0.518052663504077
25 -0.530357638486194
26 -0.529868686159328
27 -0.535633799289719
28 -0.520094384250462
29 -0.538219225309603
30 -0.504974904248363
31 -0.551215457465832
32 -0.520381891353455
33 -0.519068458932626
34 -0.485393769350976
35 -0.53016129797144
36 -0.516153650808239
37 -0.498613050658296
38 -0.4824556640531
39 -0.527296142277431
40 -0.545042782054939
41 -0.512671822677022
42 -0.527788196626389
43 -0.519345103448298
44 -0.54897334910682
45 -0.517777222087293
46 -0.520701983757483
47 -0.527497826168129
48 -0.518281350241702
49 -0.533934448312889
50 -0.49671117098909
51 -0.498206337919312
52 -0.511808216542553
53 -0.525860192675229
54 -0.524275658835744
55 -0.515416475671504
56 -0.496046383160542
57 -0.499351355977682
58 -0.517289276988344
59 -0.509398867772053
60 -0.503193358595839
61 -0.504649513219932
62 -0.511980600216809
63 -0.533281193331752
64 -0.518829651369208
65 -0.510285243954078
66 -0.512955956090999
67 -0.502851426871928
68 -0.506165489888352
69 -0.518422713748254
70 -0.507245181577103
71 -0.501062722505595
72 -0.508893458059906
73 -0.527241989235153
74 -0.49430115248243
75 -0.50930805756879
76 -0.525321612044479
77 -0.524875556367765
78 -0.53634332184394
79 -0.526024892074639
80 -0.518465759938843
81 -0.519804323474769
82 -0.512206438904766
83 -0.541924721727628
84 -0.517808629809293
85 -0.545149835159942
86 -0.527366244267706
87 -0.538937641860843
88 -0.525845788752178
89 -0.532704266807369
90 -0.515913759333647
91 -0.494506202644585
92 -0.512581052231226
93 -0.500147880919684
94 -0.526089518366295
95 -0.54773779378966
96 -0.539911625673208
97 -0.518585283998888
98 -0.511421853594712
99 -0.529715416740039
100 -0.506350455692849
};
\addplot [semithick, white!49.80392156862745!black]
table {%
0 -0.0347537853218837
1 0.183624581473794
2 0.209110157575284
3 0.194230498899752
4 0.186257421802529
5 0.145263062926003
6 0.106963807201828
7 0.0905878180190138
8 0.0845078614545053
9 0.0548479376983592
10 0.0661648660619985
11 0.0602368140697549
12 0.0716342168617151
13 0.0807286357929697
14 0.0926433559116412
15 0.0904322172167537
16 0.10112773087537
17 0.105250490637454
18 0.0839372234132653
19 0.093828417271584
20 0.0837248088984835
21 0.0501039154901367
22 0.0508859974101611
23 0.0361376302345817
24 0.0411354842163167
25 0.0321593408701002
26 0.0337995904717962
27 0.0200254692084163
28 0.00868409495801267
29 0.0284106425490875
30 0.0176819570203099
31 -2.02673589274342e-05
32 0.0207827965986463
33 0.0172694450263831
34 0.0315423258437297
35 0.0143090575579363
36 0.0311502523375126
37 0.0262018289555316
38 0.0130498785908153
39 0.0236464551622127
40 -0.00127683849995692
41 0.0228547570034927
42 0.0199121909089026
43 0.0178615163095694
44 0.0204629817262657
45 0.0040545286191706
46 -0.00120755146635757
47 0.0257138611971251
48 0.0188884632385609
49 0.0143796634090804
50 0.0134489813796692
51 0.0153393670474798
52 0.0191534754083616
53 0.00635806605080996
54 0.021003459959069
55 0.0214941604701174
56 0.0235189561343066
57 0.0111602927149601
58 0.0396292627061008
59 0.0131516055625747
60 0.030772765878525
61 0.0109396116901833
62 0.00906381010514631
63 0.0156387623303937
64 0.0158276588763597
65 0.0188254141917672
66 0.0271572396722022
67 0.0143593000019139
68 0.0385786053929
69 0.0343204349934597
70 0.00520426434859388
71 0.0213282527977477
72 0.0245392742875184
73 0.0228346024846282
74 0.0141615060271407
75 0.0107600167664541
76 0.00436077343851321
77 0.00986582027797338
78 0.0115274426683241
79 0.0199870948058446
80 0.0276847381751405
81 0.0213056608270933
82 0.0191893561364424
83 0.00244323085334764
84 0.014530839770784
85 0.010914259318541
86 0.0133450328027493
87 0.012296279105674
88 0.0111294523598638
89 0.00809852112445885
90 0.033712612691508
91 0.0120823968374935
92 0.0108743506641453
93 0.0277715840454534
94 0.0163266664684038
95 0.0103804179663015
96 0.000107494328662487
97 0.0138971053273095
98 0.0194311032349298
99 0.00616226526303811
100 0.00685024934824439
};
\addplot [semithick, color7]
table {%
0 0.103524821653907
1 0.544086093922144
2 0.639031592394743
3 0.622371134760811
4 0.646414146133206
5 0.613463535334476
6 0.649074943314145
7 0.643163871603883
8 0.650996936764474
9 0.596164869891439
10 0.602538462038016
11 0.548636156477343
12 0.505898452302246
13 0.448722185843826
14 0.395568047543233
15 0.318691415331093
16 0.272908993491909
17 0.222486429949094
18 0.163720449393024
19 0.148573585454597
20 0.123263658388318
21 0.0809669329873423
22 0.0737375504782896
23 0.0562096303908364
24 0.0599670753508323
25 0.0479000253757225
26 0.0491941373918737
27 0.033841294794756
28 0.0229008686517991
29 0.0409784791556435
30 0.0303312510806218
31 0.0112076187409145
32 0.0319146569025088
33 0.0256035665247917
34 0.0386489542805878
35 0.0209408150341246
36 0.0371089073337566
37 0.0303203826496743
38 0.0164141318206579
39 0.0260735775654926
40 0.000850282332299679
41 0.0247337080391569
42 0.0213595014265164
43 0.0193838123784954
44 0.0214698343064323
45 0.004969856922184
46 -0.000329440965472374
47 0.0263431052678105
48 0.0195015089095617
49 0.0149312406323932
50 0.0139670086996453
51 0.0158111226072478
52 0.0194439217184697
53 0.00659033398458689
54 0.0212723243387683
55 0.0216956452331671
56 0.0236890069723382
57 0.0112955770165069
58 0.0397587923325353
59 0.0132747207855661
60 0.0308654553158957
61 0.0110261351833114
62 0.00914935563317996
63 0.0157261289504271
64 0.0158918431670034
65 0.0188676076737901
66 0.027196132110917
67 0.0143985391217913
68 0.0386187609133504
69 0.0343500890676036
70 0.00523269835253656
71 0.0213474379133895
72 0.0245559307328908
73 0.0228459042411395
74 0.0141717926511686
75 0.0107698590097392
76 0.00437093196396537
77 0.00987383803252674
78 0.0115339844686899
79 0.0199923019609136
80 0.0276895548460159
81 0.0213111281585455
82 0.0191938260805381
83 0.00244634526032149
84 0.0145332552122932
85 0.0109166571194233
86 0.0133471156662269
87 0.0122981194699526
88 0.0111310947606122
89 0.00809999064908244
90 0.0337143055079896
91 0.0120838525583662
92 0.0108755161323533
93 0.0277718538322616
94 0.0163269266293071
95 0.0103806003800917
96 0.000107817081267117
97 0.0138972356541625
98 0.0194313352975268
99 0.00616249160998873
100 0.00685042060374245
};
\addplot [semithick, color8]
table {%
0 -0.00316425560159981
1 -0.551884164559504
2 -0.586489265761776
3 -0.590645980927423
4 -0.588248253783337
5 -0.562561009418434
6 -0.599021525753339
7 -0.58723218274946
8 -0.591874520915488
9 -0.600573353526525
10 -0.633847130691818
11 -0.663273429556351
12 -0.688082709932677
13 -0.730233521074813
14 -0.779041805920324
15 -0.811021483991179
16 -0.848894010332733
17 -0.847473931067515
18 -0.84146258494449
19 -0.861487571780675
20 -0.876056939447075
21 -0.905962821208245
22 -0.892593263743505
23 -0.882684495370136
24 -0.875644108835224
25 -0.916917674036718
26 -0.895046427115017
27 -0.907633430284799
28 -0.906715746340089
29 -0.917156992980821
30 -0.915034844739957
31 -0.915104856090206
32 -0.913304512102069
33 -0.910003207501194
34 -0.910093044754453
35 -0.918535923739974
36 -0.905646974059089
37 -0.914202214082676
38 -0.904989565495009
39 -0.909216992908494
40 -0.913374566470597
41 -0.918156737505013
42 -0.898415819567806
43 -0.900867549061816
44 -0.923744518854978
45 -0.895461375458451
46 -0.91016146456347
47 -0.908598882505409
48 -0.902968201272389
49 -0.928837270187323
50 -0.901499290010776
51 -0.90088312147337
52 -0.894607276299297
53 -0.931857994683617
54 -0.917887136636071
55 -0.902978454453475
56 -0.911417551184014
57 -0.915203616375598
58 -0.896372446419839
59 -0.911067965423337
60 -0.89695957182203
61 -0.896640002163532
62 -0.918561744995528
63 -0.91557802652936
64 -0.922399322145042
65 -0.90053179930625
66 -0.910750752761899
67 -0.904151781379141
68 -0.91925938276107
69 -0.90533603833553
70 -0.919524378679991
71 -0.928430689404262
72 -0.904299261931158
73 -0.92622226842836
74 -0.871505166895064
75 -0.899503098630051
76 -0.890285310573249
77 -0.91530456247664
78 -0.924870893241424
79 -0.871808328708295
80 -0.894884444905939
81 -0.897909008455783
82 -0.898390952600419
83 -0.908653274898454
84 -0.901981424107316
85 -0.911494295295591
86 -0.918397273187851
87 -0.923316902894916
88 -0.909217551234081
89 -0.893229152678968
90 -0.891473750888131
91 -0.894668739807541
92 -0.894569130529349
93 -0.899065333263439
94 -0.913162157631275
95 -0.905573360457216
96 -0.914882249864423
97 -0.887002805692941
98 -0.896354447338597
99 -0.893047598875599
100 -0.878658647281632
};
\addplot [semithick, color0]
table {%
0 -0.103878237081972
1 -0.232390324319672
2 -0.718961211824001
3 -0.724703798907217
4 -0.790877771642585
5 -0.779331643706204
6 -0.819560078524139
7 -0.85879763417778
8 -0.880698829373831
9 -0.852099876040054
10 -0.882386704666423
11 -0.882133770810676
12 -0.896649195700068
13 -0.881608229363474
14 -0.895132843294651
15 -0.904000536285391
16 -0.889599513661173
17 -0.887786720992731
18 -0.897407085793158
19 -0.902924074022581
20 -0.897908691102918
21 -0.908916254876674
22 -0.919494822972225
23 -0.905545305668702
24 -0.891786773580539
25 -0.918630585513304
26 -0.919663154922482
27 -0.915167536390637
28 -0.908006431011412
29 -0.894190085938902
30 -0.921847978527732
31 -0.898142513861618
32 -0.886315259413001
33 -0.901126795363205
34 -0.91799400958555
35 -0.935071108322662
36 -0.927631538736596
37 -0.898597237029383
38 -0.914435753338393
39 -0.894457724517154
40 -0.928800945120382
41 -0.930297720710369
42 -0.887326139753893
43 -0.901909749911843
44 -0.939392586094912
45 -0.902388418339918
46 -0.912362069449088
47 -0.893981696653263
48 -0.89730142114225
49 -0.916694328404271
50 -0.903219797970274
51 -0.884810677503678
52 -0.908473410094503
53 -0.922157475985991
54 -0.921030049982711
55 -0.901663287891288
56 -0.906115316771897
57 -0.904815415055439
58 -0.901585301290125
59 -0.902559378063257
60 -0.904927803636197
61 -0.899143392758159
62 -0.901415146919706
63 -0.94037997165469
64 -0.91639845968435
65 -0.924181622263633
66 -0.925116609956074
67 -0.932198261437891
68 -0.927050130092876
69 -0.924105881230504
70 -0.92501251008614
71 -0.913008453907018
72 -0.909473077209824
73 -0.919906771966434
74 -0.909520579855912
75 -0.905866649287088
76 -0.896943733798557
77 -0.898857534477348
78 -0.89941951419347
79 -0.89611172960982
80 -0.882831485580362
81 -0.902355349270266
82 -0.897501249721723
83 -0.892050618163546
84 -0.891170039803858
85 -0.881662649411437
86 -0.906335428148532
87 -0.920308021836149
88 -0.924610150936021
89 -0.90281605566994
90 -0.900806787221913
91 -0.893784902904567
92 -0.915780656718415
93 -0.906518026061978
94 -0.885785395157491
95 -0.914065974050405
96 -0.889590616016484
97 -0.912481061113688
98 -0.919516415619312
99 -0.914975393983251
100 -0.918998933040513
};
\addplot [semithick, color1]
table {%
0 -0.0191880164680057
1 0.209971645165173
2 0.484149781423604
3 0.46854143677007
4 0.452437256147545
5 0.335452410229626
6 0.303127807168031
7 0.206269604505201
8 0.111329555887237
9 0.0553487474622991
10 0.00607303986781844
11 -0.0205237527615726
12 -0.0430425240786252
13 -0.0504252364466452
14 -0.0263267910041342
15 -0.0372870762843757
16 -0.0369660225752434
17 -0.0328788507167061
18 -0.0289496517893605
19 -0.0219267218885924
20 -0.0394574457243829
21 -0.0220646432952326
22 -0.0267976350078749
23 -0.0228928735507559
24 0.00171881417575093
25 -0.00345020829289528
26 -0.00241565307432019
27 -0.00265358312850983
28 -0.00967765265422484
29 0.00458481485512225
30 -0.00762506448969655
31 -0.00445402341169365
32 0.0116606264380779
33 -0.0172554119010825
34 0.0254745750385604
35 0.00297724960387549
36 0.0015528616388671
37 0.0161375732438681
38 -0.00225792101483604
39 0.00728996334519941
40 0.00267559827682144
41 -0.0155750350380517
42 0.0156783025911881
43 0.00934887182873951
44 0.00642386574711392
45 0.0115631422413336
46 -0.0238273348684861
47 0.00878122836595701
48 0.00272933076949852
49 0.0374881791804901
50 -0.00131752877236259
51 0.0174923131558636
52 0.00318092772870632
53 -0.0111200427268383
54 0.0090242244911024
55 -0.00346653455935217
56 -0.00534102246790137
57 0.0133392314738155
58 0.00710634726823945
59 0.0286653637028002
60 0.00279739431376252
61 0.000161346051324444
62 0.008151014115203
63 -0.00200253191548476
64 0.00559442226064534
65 -0.00557185782374148
66 0.00498295129097135
67 -0.00231836437801599
68 0.000971238423934319
69 -0.00858336391232978
70 -0.0050997829584395
71 0.00212870860185269
72 0.0172083610938994
73 0.00908632676097891
74 -0.00964843873882453
75 -0.0124169528622357
76 0.011026410539083
77 0.00501100174691696
78 0.0152337008019174
79 0.00118077116786463
80 0.00597389198226116
81 0.000861722837167197
82 0.00463583078969008
83 0.0132724290338852
84 -0.0107234904928183
85 0.020760063052216
86 -0.00447635591530576
87 -0.0101589279568676
88 0.00156303002366317
89 0.0202522944593328
90 0.0139782883094433
91 0.00465016778816871
92 0.00603362999655043
93 0.000176764802447763
94 0.0156397813367334
95 -0.0120748796556602
96 0.00576238054903603
97 0.00249680575327825
98 -0.00113777946003825
99 0.00123384699443918
100 -0.00541588366899399
};
\addplot [semithick, color2]
table {%
0 0.0345728655263056
1 0.0796285308432766
2 -0.252472990845532
3 -0.224904109737
4 -0.222792691388055
5 -0.199497988923426
6 -0.144192072009196
7 -0.103313392398611
8 -0.092738064860335
9 -0.0507660200822544
10 -0.0487969176986121
11 -0.0455871361252492
12 -0.0158593985870557
13 -0.0301819072631852
14 0.0130536913330842
15 0.00384754228542239
16 0.0139154409083454
17 0.00731753398741136
18 0.0117406566894699
19 0.0176053789972986
20 0.000857559091275617
21 0.0139680023839463
22 0.00793428032620156
23 0.00984689273477355
24 0.0281457650345918
25 0.0210467239008867
26 0.0221464982221879
27 0.0202605759814589
28 0.0117562674211763
29 0.0235906181186308
30 0.01127984776278
31 0.0145632150099159
32 0.0270698330009901
33 -0.00404203166966326
34 0.036679480271007
35 0.0124305668394
36 0.0104180524966458
37 0.0228015583609217
38 0.00325023431549793
39 0.0117750866907529
40 0.00669730085501724
41 -0.0119365419454141
42 0.0190813584161928
43 0.0125514623313196
44 0.0091645684205286
45 0.0137572188752709
46 -0.021968927256285
47 0.0103700789481699
48 0.00416100573894149
49 0.038709445571876
50 -0.000266253313947977
51 0.0184242745613534
52 0.003875773088009
53 -0.0104951441301483
54 0.00956115486115977
55 -0.00302520091656722
56 -0.0049461984247521
57 0.0136554407126414
58 0.00737335371760874
59 0.0288906409946457
60 0.00297697976155669
61 0.000324551430321353
62 0.00829180327763133
63 -0.0018706072399967
64 0.00570120097796705
65 -0.00548432312417803
66 0.00505761908336923
67 -0.00225545862257369
68 0.00102316183304896
69 -0.00853272753277347
70 -0.00505003438042297
71 0.00217716253031766
72 0.0172547757043295
73 0.00912492097114204
74 -0.0096147612298713
75 -0.012389853485852
76 0.0110508979289331
77 0.00503245330142201
78 0.0152507696219004
79 0.00119680181811119
80 0.00598793605840513
81 0.000875190904919735
82 0.00464863480742056
83 0.0132830343223475
84 -0.0107136536683259
85 0.0207687232924269
86 -0.00446943782580401
87 -0.0101525069319948
88 0.00156912282511446
89 0.0202574548828064
90 0.0139826844576363
91 0.00465414682452703
92 0.00603707646114739
93 0.000179558090464631
94 0.015642046102991
95 -0.0120728349457816
96 0.00576421233058171
97 0.00249841929200443
98 -0.00113639612034825
99 0.00123505458128434
100 -0.00541491519374103
};
\addplot [semithick, color3]
table {%
0 0.0471269017190401
1 0.134714905377992
2 0.338245431469573
3 0.366344459390395
4 0.400357470702474
5 0.40141741473889
6 0.483600084511999
7 0.51218977738979
8 0.548022533149238
9 0.567955944289624
10 0.560929305018825
11 0.557428966565616
12 0.521250977940695
13 0.544604746249315
14 0.561660416525788
15 0.553825942851165
16 0.511736487432431
17 0.545802858545414
18 0.536392495549649
19 0.543544343671032
20 0.532796321213115
21 0.551465382985283
22 0.548253063914547
23 0.524472514926024
24 0.537165323413802
25 0.535977792176642
26 0.505473995020787
27 0.512536629758853
28 0.511083379269382
29 0.528800683310753
30 0.524174212105777
31 0.506312229966475
32 0.497002713903754
33 0.494713367279141
34 0.540536900424047
35 0.50130163109238
36 0.510778248789829
37 0.530135017185553
38 0.512313047199377
39 0.517453175070086
40 0.563669076399094
41 0.52882118852669
42 0.520647201362445
43 0.527721477483918
44 0.506966440133378
45 0.54164002041811
46 0.493087933604979
47 0.525319727542619
48 0.497460910635036
49 0.576941958759903
50 0.517023209732087
51 0.534655480538147
52 0.502793084461666
53 0.528526992060316
54 0.5203894237944
55 0.514397530175834
56 0.533187009243813
57 0.522353979421195
58 0.485542469779377
59 0.518003607622977
60 0.50470353156821
61 0.522172939587349
62 0.533327844334023
63 0.49696525599827
64 0.503765435397471
65 0.494272457757167
66 0.515800854861813
67 0.504410098907675
68 0.496116737639482
69 0.488676294505559
70 0.501776662366401
71 0.51461025142653
72 0.52461576759208
73 0.519207559722327
74 0.519313663656078
75 0.519789228599514
76 0.496864404439549
77 0.547891169130817
78 0.539674211609635
79 0.516763724286291
80 0.480036819432306
81 0.501660718893655
82 0.539339029707554
83 0.51564461148642
84 0.508624267806847
85 0.518910184451764
86 0.534709437914655
87 0.50688274696306
88 0.529792302917266
89 0.534351409307872
90 0.49911798848937
91 0.514013709560963
92 0.507210083675839
93 0.523371546619666
94 0.52721304819056
95 0.548977362827468
96 0.527503667065944
97 0.530944677067998
98 0.507018264599302
99 0.495376900842896
100 0.501593017965815
};
\end{groupplot}

\end{tikzpicture}
%     }
%     \caption{\label{fig:QD-rbm-training}Left: Performance of the
%       wave function in ?? as a function of
%       training steps. Right: Progression of variational parameters as a function
%       of training steps. The source code for this graphic can be found~\cite[TODO: Add
%     path]{MS-thesis-repository}, and \LaTeX{} output generated
%     by~\cite{nico_schlomer_2018_1173090}.}
% \end{figure}

% \begin{table}[h]
%   \centering
%   \begin{tabular}{lS[table-format=1.6]*2{S[table-format=1.4]}*2{S[table-format=1.1]}}
\toprule
\addlinespace
& {$\langle E_L\rangle$} & {CI$^{95}_-$} & {CI$^{95}_+$} & {Std} & {Var} \\
\addlinespace
\midrule
\addlinespace
\addlinespace
    $\psi_{RBM}$ & 3.0796(9) & 3.0779 & 3.0813 & \num{1.1e-01} & \num{1.3e-02}\\
$\psi_{SRBM}$ & 3.0805(9) & 3.0788 & 3.0822 & \num{1.6e-01} & \num{2.5e-02}\\
\addlinespace\addlinespace\bottomrule
\end{tabular}
%   \caption{Energy using RBM wave function, using $2^{22}$
%     Monte Carlo samples and errors estimated by an automated blocking algorithm
%     by~\textcite{Jonsson-2018}. See \cref{fig:QD-benchmark-pade-jastrow-training}
%     for source code reference.}
%   \label{tab:rbm-energy-results}
% \end{table}

% \begin{figure}[h]
%    \centering
%     \resizebox{0.7\linewidth}{!}{%
%         % This file was created by matplotlib2tikz v0.7.4.
\begin{tikzpicture}

\definecolor{color0}{rgb}{0.12156862745098,0.466666666666667,0.705882352941177}

\begin{groupplot}[group style={group size=2 by 1}]
\nextgroupplot[
legend cell align={left},
legend style={at={(0.97,0.03)}, anchor=south east, draw=white!80.0!black},
log basis x={10},
tick align=outside,
tick pos=left,
x grid style={white!69.01960784313725!black},
xlabel={\% of training},
xmin=0.794328234724281, xmax=125.892541179417,
xmode=log,
xtick style={color=black},
y grid style={white!69.01960784313725!black},
ylabel={Symmetry},
ymin=0.733676636785089, ymax=1.00454507289283,
ytick style={color=black}
]
\addplot [semithick, color0]
table {%
0 0.790505014204757
1 0.74598883842635
2 0.947768626070734
3 0.96948585133299
4 0.953696161708256
5 0.974937535253731
6 0.923344838956568
7 0.961482390384219
8 0.991393678183755
9 0.95029834203381
10 0.968925449589554
11 0.973630135858377
12 0.992232871251569
13 0.953558611454892
14 0.985888140372131
15 0.950838015580814
16 0.93791465622522
17 0.95835758588659
18 0.968183418266356
19 0.956969899949039
20 0.942333975059768
21 0.97157501572833
22 0.98580685043472
23 0.967441527220482
24 0.945763019878873
25 0.97460994353409
26 0.971987007897458
27 0.937316101891023
28 0.935238287167064
29 0.933694541032157
30 0.965310479375889
31 0.944045071254866
32 0.955806569587109
33 0.961508700360241
34 0.935313991456203
35 0.959514617857582
36 0.966280710260126
37 0.958139903836819
38 0.965017913583062
39 0.963646250170808
40 0.967674090491502
41 0.942293178571695
42 0.938458132136521
43 0.975323418485579
44 0.977440884525939
45 0.947431836260294
46 0.913987664591805
47 0.962302293515426
48 0.991748500907372
49 0.97285106186465
50 0.973496333479417
51 0.948597087806652
52 0.957676937373142
53 0.939485602018924
54 0.961202696006406
55 0.970260153356246
56 0.96050079371464
57 0.984889139044278
58 0.95107998902375
59 0.980347826092903
60 0.955745940340491
61 0.973478805682525
62 0.956126230642975
63 0.939742158824222
64 0.983032826271889
65 0.932666397142107
66 0.971488278680402
67 0.969132321280105
68 0.953119607137017
69 0.922394178520304
70 0.980804190919468
71 0.968783271629619
72 0.945981595966271
73 0.977363533935965
74 0.970493986303562
75 0.99117602699472
76 0.96780761923677
77 0.98084878291549
78 0.964374093479722
79 0.960498112205791
80 0.940683265043989
81 0.955744454447226
82 0.988301964687945
83 0.967283072501018
84 0.989478002859061
85 0.95994144311729
86 0.970175653743393
87 0.939796153233789
88 0.974720645914907
89 0.925993315248282
90 0.95644528600371
91 0.984558908023625
92 0.950430111210129
93 0.961831636148792
94 0.973643080318876
95 0.979030782836415
96 0.933867010699962
97 0.948936149914941
98 0.980843644325301
99 0.948853823790807
100 0.966173499904257
};
\addlegendentry{$S(\psi_{RBM})$}

\nextgroupplot[
colorbar,
colormap/viridis,
point meta min=-1,
point meta max=1,
tick align=outside,
x grid style={white!69.01960784313725!black},
xlabel={\(\displaystyle \mathbf{W}\)},
xmin=-0.5, xmax=3.5,
xtick pos=both,
xtick style={color=black},
y grid style={white!69.01960784313725!black},
ymin=-0.5, ymax=3.5,
ytick pos=left,
ytick style={color=black}
]
\addplot graphics [includegraphics cmd=\pgfimage,xmin=-0.5, xmax=3.5, ymin=3.5, ymax=-0.5] {scripts/QD-rbm.py.symmetry-000.png};
\end{groupplot}

\end{tikzpicture}
%     }
%     \caption{\label{fig:QD-rbm-symmetry}Permutation symmetry of the RBM wave
% function in ?? as a function of training steps. The source code for this graphic
% can be found~\cite[TODO: Add path]{MS-thesis-repository}, and \LaTeX{} output
% generated by~\cite{nico_schlomer_2018_1173090}.}
% \end{figure}

An interesting consideration that arises from this form of wave function model
is that it has no guarantee of satisfying the required permutation symmetry.
This is an attribute of most neural network based models. Because we know the
true ground state must have the correct symmetry, we would hope that the RBM is
able to realize that a symmetric form is best. To this end we have defined a
metric $S(\psi)$, which has the property of being equal to 1 for symmetric
functions and $0$ for anti-symmetric ones. See \cref{app:symmetry-metric} for
its definition and details. The left plot in \cref{fig:QD-rbm-symmetry} shows a
plot of the symmetry metric of $\psi_{RBM}$ during training. Luckily we see
that the RBM, which starts completely non-symmetric, tends rapidly towards being
fully symmetric. Still it is not purely symmetric, and we see slight
oscillations around the maximum value.

The right plot in \cref{fig:QD-rbm-symmetry} shows a peculiar pattern emerging
from the weights $\mat W$ of the final model. Several weights are
inconsequential, while the remaining come in pairs of equal value. Additionally,
although not obvious from the plot, there are only two unique absolute values,
i.e. $\pm v_1$ and $\pm v_2$. Increasing the number of hidden nodes results in
the same two strips, with the extra elements similarly zeroing out. 

In attempts to properly fix the symmetry of $\psi_{RBM}$, we have made two meaningful
attempts. The first was setting $\vb a =a \vb 1, \vb b = b\vb 1$, i.e. removing
different biases for different degrees of freedom, and reducing the weights to
$\mat W\in\mathbb{R}^{D\times N}$. The motivation was that this would ensure all
degrees of freedom was treated in a symmetric manner. While this ensured
$S(\psi_{RBM})=1$, the resulting wave function proved unable to learn anything
useful. It could still learn the ground state if the particles where
non-interacting (i.e. the pure Gaussian), but failed remarkably in the full
system.

More successfully we imposed symmetry by sorting the inputs prior to feeding
them through $\psi_{RBM}$. While this was able to learn something, it stalled
out at $>\SI{3.1}{\au}$. In general it is our experience that imposing the
symmetry from the beginning of the training stops the RBM in getting anywhere.
This appears to be a manifestation of the classical problem in learning of balancing
exploitation and exploration, with this case suffering from to little
exploration.

The only truly successful way we found was to train the RBM as before, and then
apply sorted inputs on it once it had been fully trained. This lead to only a
marginal increase in energy.

\section{Neural Network}

\end{document}
