\documentclass[Thesis.tex]{subfiles}
\begin{document}
\chapter{Quantum Dots}
\label{chp:quantum-dots}

We now present results for all methods discussed, applied on the example system
of Quantum Dots (QD) from \cref{sec:quantum-dots-theory}. We present first some
benchmarks using a typical Slater-Jastrow wave function form, followed by
introduction of various neural network-based wave function.

\section{Benchmark}

As we shall restrict this analysis to only two interacting particles in the QD,
$\vX = (\vx_1\ \vx_2)$, our benchmark wave function is rather simple. We build
it up using the product of single particle ground states (Gaussian), multiplied
by a Pade-Jastrow correlation term\footnote{We drop the constant factor
  from~\cref{eq:Phi-non-inter} because we have not normalized the wave function.}:

\begin{align}
  \label{eq:qd-pade-jastrow-anzats}
  \psi_{PJ}(\vX) &= \Phi(\vX) \,J_P(\vX)\\
  &= \exp(-\alpha_G\sum_{i=1}^N \norm{\vx_i}^2 + \sum_{i < j} \frac{\alpha_{PJ}
    r_{ij}}{1 + \beta_{PJ} r_{ij}}),
\end{align}
where $\alpha_{PJ} = 1$ is fixed by the cusp conditions, and $\alpha_G$ and $\beta_{PJ}$
are the two only variational parameters.

\subsubsection{Optimizing}

We have run a simple optimization of the above wave function, using initial
values of $\alpha_G = 0.5$ and $\beta_{PJ} = 1$. We used importance sampling and the
ADAM optimization scheme. We used $\num{2000}$ optimization steps, each with
$\num{5000}$ MC cycles. The values are somewhat arbitrarily, and we get
similar results for several other choices.

\begin{figure}[h]
   \centering
    \resizebox{\linewidth}{!}{%
        % This file was created by matplotlib2tikz v0.7.4.
\begin{tikzpicture}

\definecolor{color0}{rgb}{0.12156862745098,0.466666666666667,0.705882352941177}
\definecolor{color1}{rgb}{1,0.498039215686275,0.0549019607843137}

\begin{groupplot}[group style={group size=2 by 1}]
\nextgroupplot[
legend cell align={left},
legend style={draw=white!80.0!black},
tick align=outside,
tick pos=left,
x grid style={white!69.01960784313725!black},
xlabel={\% of training},
ylabel={Ground state energy $[\si{\au}]$},
xmin=-5, xmax=105,
xtick style={color=black},
y grid style={white!69.01960784313725!black},
ymin=2.99704592819679, ymax=3.0553945154196,
ytick style={color=black},
ylabel near ticks,
]
\addplot [semithick, color0]
table {%
0 3.05274230690948
1 3.03793728796287
2 3.03042932078306
3 3.03147832063556
4 3.02844507193604
5 3.02088514786338
6 3.02511840035016
7 3.01862389645188
8 3.01922001846441
9 3.02448078551538
10 3.01695374152415
11 3.02149449163837
12 3.01751935807795
13 3.01271928794601
14 3.01647427586709
15 3.00939605941487
16 3.01386722338982
17 3.01304696248959
18 3.00992166949493
19 3.00892353607615
20 3.01016826114483
21 3.01039088846332
22 3.00426445580458
23 3.00645324289482
24 3.00501888357255
25 3.00348186918386
26 3.00441720089552
27 3.00442281302622
28 3.0018750604523
29 3.00355148586118
30 3.00171137956311
31 3.00298038895463
32 3.00080951818635
33 3.00105332554284
34 3.00195265921159
35 3.0005461422342
36 2.9999000956281
37 3.00019076693779
38 3.00076928146752
39 3.00042162249406
40 3.00135401658992
41 3.00090852610597
42 3.0007619299255
43 3.00072168045281
44 3.00064448372425
45 3.00040557287456
46 3.00111714273009
47 3.00076544340293
48 3.00035070540922
49 3.00092541277394
50 3.00022734746425
51 3.00046526568344
52 3.00059781409551
53 3.00063381420564
54 3.00035451805525
55 3.00037117495291
56 3.00134892692779
57 3.00075190853335
58 3.00065477638969
59 3.00021139892498
60 3.00029658491584
61 3.00049936563556
62 3.00049566724919
63 3.00063981769536
64 3.00062485027585
65 3.00080855104683
66 3.00035673435316
67 3.00068375879364
68 3.0008476157282
69 3.00092963123955
70 3.00071206112081
71 3.00038036745322
72 3.00059580814765
73 3.00064194367538
74 3.00047228782548
75 3.00051056917616
76 3.00018337005192
77 3.00031215610914
78 3.00084542377648
79 3.00138368193253
80 3.00130159778628
81 3.00073305314236
82 3.00025821935658
83 3.00029180410207
84 3.00022019576071
85 3.00053189012777
86 3.00050436447339
87 3.00075486546648
88 3.00072817806753
89 3.0011932402123
90 3.00098661658882
91 3.00107506850719
92 3.00025494563264
93 2.999880129195
94 2.99969813670692
95 3.00049469036751
96 3.00141903905299
97 3.00018861498403
98 3.00065423672549
99 3.0002256484428
100 3.000608740078
};
\addlegendentry{$\psi_{PJ}$}
\addplot [semithick, black, opacity=0.5, dashed]
table {%
-5 3
105 3
};
\addlegendentry{Exact}

\nextgroupplot[
legend cell align={left},
legend style={draw=white!80.0!black},
tick align=outside,
tick pos=left,
x grid style={white!69.01960784313725!black},
xlabel={\% of training},
ylabel={Parameters},
xmin=-5, xmax=105,
xtick style={color=black},
y grid style={white!69.01960784313725!black},
ymin=0.366969287310357, ymax=1.03014431965189,
ytick style={color=black},
ylabel near ticks,
ytick pos=right,
yticklabel pos=right,
ylabel style={rotate=-180},
]
\addplot [semithick, color0]
table {%
0 0.5
1 0.480655926967057
2 0.463697501976467
3 0.451192127077001
4 0.442737507443148
5 0.438111608232936
6 0.436809678602654
7 0.439550227846682
8 0.440370661362374
9 0.44196620588902
10 0.441699815490073
11 0.441673424505602
12 0.442794615290762
13 0.445288540681495
14 0.44524344435607
15 0.44599998860998
16 0.449872276110169
17 0.451396959301731
18 0.454260819395208
19 0.455092374589909
20 0.456843530760932
21 0.458709500679744
22 0.460615876660145
23 0.460374712476665
24 0.463261215859744
25 0.464949617594546
26 0.465163534320254
27 0.469088094313409
28 0.473001222172205
29 0.473886018788919
30 0.474397605060466
31 0.4766342480579
32 0.479277060460206
33 0.481446936562426
34 0.482341299014396
35 0.483129614027315
36 0.485187340832105
37 0.485484833442987
38 0.486849156774872
39 0.488539929624974
40 0.489226179672706
41 0.49061368339717
42 0.491287205504628
43 0.49121271170576
44 0.491618280745601
45 0.491856226891644
46 0.49209571237464
47 0.493312673786748
48 0.493344650527207
49 0.493349972324268
50 0.493554335750895
51 0.493742625998036
52 0.493684536935949
53 0.493418273715207
54 0.493663912511312
55 0.493795231166937
56 0.494313551308408
57 0.494756080546951
58 0.49396689406379
59 0.49461122060446
60 0.494502925522114
61 0.494288988911376
62 0.494628077475571
63 0.49413752152439
64 0.49414972531498
65 0.493971120843915
66 0.494164409824097
67 0.493759655306835
68 0.494003530172205
69 0.494028572113906
70 0.494729176763253
71 0.494287778501983
72 0.493450583197636
73 0.493995142627925
74 0.493989208145016
75 0.493880220631274
76 0.493849266353634
77 0.494615161941479
78 0.494383290905142
79 0.49392515784999
80 0.49490124712869
81 0.494709209883818
82 0.494705592833194
83 0.49453674993233
84 0.494268510488658
85 0.493701388063222
86 0.494591361060595
87 0.494108037162981
88 0.494689893738149
89 0.494466558744206
90 0.493822377138241
91 0.494849593582664
92 0.494919036209166
93 0.493659689615945
94 0.493698639179818
95 0.494660104397708
96 0.494601257652416
97 0.494625829393491
98 0.494284436275973
99 0.494189231282311
100 0.49501908671325
};
\addlegendentry{$\alpha_G$}
\addplot [semithick, color1]
table {%
0 1
1 0.980141471446477
2 0.961154099549839
3 0.94354541962038
4 0.926263381376864
5 0.909188306126108
6 0.89229994851214
7 0.875877528367407
8 0.859278833730977
9 0.842864550195302
10 0.825497233124107
11 0.807977785668522
12 0.791502880092063
13 0.774335581707413
14 0.756734655277644
15 0.739382782996785
16 0.722002653039462
17 0.704091868819612
18 0.685961062328723
19 0.668534931746746
20 0.65201323793735
21 0.634798789489699
22 0.618312733285203
23 0.601305182751218
24 0.58601777009998
25 0.571094569781092
26 0.556726229571795
27 0.543178962981069
28 0.529220850286612
29 0.515032109619336
30 0.502466196498837
31 0.49118036743747
32 0.480106406224218
33 0.469854110847922
34 0.459905925210542
35 0.45169073074283
36 0.444085166937463
37 0.436721490015063
38 0.43086027216831
39 0.426019836452593
40 0.421567023051503
41 0.418175341140745
42 0.415147180000562
43 0.412114996741048
44 0.409742178054753
45 0.407544904686301
46 0.406108097925919
47 0.405401962226926
48 0.404109574090475
49 0.403000552712037
50 0.402386691318389
51 0.40198936056923
52 0.401304016882003
53 0.400463395040488
54 0.399907486357164
55 0.399927548840752
56 0.399439318254555
57 0.399230167045008
58 0.398594625417332
59 0.398836667613015
60 0.398714339786569
61 0.398588601560595
62 0.398655699546475
63 0.397913251765501
64 0.397897219990408
65 0.397471495679413
66 0.397552186914199
67 0.39713574127287
68 0.397113606962245
69 0.397829064337513
70 0.399062510574475
71 0.398682882777356
72 0.397791930555549
73 0.398332439546336
74 0.39840688173924
75 0.398196740761694
76 0.397787173238738
77 0.397493872059313
78 0.398208183155796
79 0.397970236915008
80 0.398404448782546
81 0.398493589328846
82 0.399081438402194
83 0.398881611564127
84 0.398476913330119
85 0.398767471837898
86 0.398649655081354
87 0.398289170410177
88 0.399078939980803
89 0.398678567980949
90 0.39764844532161
91 0.397948711272731
92 0.399085817637872
93 0.398518195003331
94 0.39858533693385
95 0.398898472417307
96 0.398514097278833
97 0.397936245658474
98 0.398139502941347
99 0.397847090314031
100 0.397793881004863
};
\addlegendentry{$\beta_{PJ}$}
\end{groupplot}

\end{tikzpicture}
    }
    \caption{\label{fig:QD-benchmark-pade-jastrow-training}Left: Performance of the
      wave function in \cref{eq:qd-pade-jastrow-anzats} as a function of
      training steps. Right: Progression of variational parameters as a function
      of training steps. The source code for this graphic can be found~\cite[TODO: Add
    path]{MS-thesis-repository}, and \LaTeX{} output generated
    by~\cite{nico_schlomer_2018_1173090}.}
\end{figure}

\begin{table}[h]
  \centering
  \begin{tabular}{lccccc}
\toprule
\addlinespace
& $\langle E_L\rangle$ & CI$^{95}_-$ & CI$^{95}_+$ & Std & Var \\
\midrule
    $\Phi$ & 3.250(2) & 3.246 & 3.254 & \num{1.8e-01} & \num{3.3e-02}\\
$\psi_{PJ}$ & 3.00066(6) & 3.00055 & 3.00077 & \num{5.0e-03} & \num{2.5e-05}\\
\bottomrule
\end{tabular}
  \caption{Energy benchmark using Pade-Jastrow wave function, using $2^{22}$
    Monte Carlo samples and errors estimated by an automated blocking algorithm
    by~\textcite{Jonsson-2018}. See \cref{fig:QD-benchmark-pade-jastrow-training}
    for source code reference.}
  \label{tab:pade-jastrow-benchmark-energy}
\end{table}

\cref{fig:QD-benchmark-pade-jastrow-training} shows the optimization as function
of percentage of training completed. We can observe that the optimizations
quickly settles down to a set of optimal values, where it only oscillates
slightly back and forth. \cref{tab:pade-jastrow-benchmark-energy} shows
statistics for the energy obtained with the final state. Comparing to the
analytical result of $\SI{3}{\au}$ these results are in very good agreement. For
reference we have also given the results obtained without the Pade-Jastrow term,
i.e. the non-interacting ground state.

\section{Restricted Boltzmann Machine}

\section{Neural Network}

\end{document}
