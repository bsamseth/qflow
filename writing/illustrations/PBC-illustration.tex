\begin{tikzpicture}
\def\proton(#1,#2){%
    \shade[ball color=red] (#1,#2) circle (3pt);
    \node at (#1,#2) {};
}
\def\neutron(#1,#2){%
    \shade[ball color=white] (#1,#2) circle (3pt);
}
\def\electron{%
    \shade[ball color=green!30] (0,0) circle (1pt);
    \node at (0,0) {};
}
\def\sorbit(#1,#2,#3,#4,#5){%
  \draw[
  color=black!30,
  postaction=decorate,
  decoration={markings,
  mark=at position #5 with {\electron},
}]
  (#1,#2) circle[x radius=#3, y radius=#4];
}

\def\helium(#1,#2,#3,#4,#5,#6){%
  \neutron(#1+0.1,#2)
  \proton(#1,#2+0.1)
  \proton(#1,#2-0.1)
  \neutron(#1-0.1,#2)
  \sorbit(#1, #2, 0.2+#5, 0.2+#6, #3)
  \sorbit(#1, #2, 0.2+#6, 0.2+#5, #4)
}

\helium(0.45,3.3,0.3,0.9,0.1,0.02)
\helium(1.6,4.4,0.1,0.5,0.1,0.02)
\helium(0.5,4.45,0.2, 0.7,0.1, 0.02)
\helium(1.1,3.7, 0.3, 0.7,0.1, 0.02)

% edge of the initial simulation box
\path[draw,ultra thick] (0,3) rectangle (2,5) node at (1,5.3) {Simulation box};


% periodic boundaries
\foreach \x in {5,7,9} {
    \foreach \y in {1,3,5} {
        \helium(\x+0.45,\y+0.3,0.3,0.9,0.1,0.02)
        \helium(\x+1.6 ,\y+1.4,0.1,0.5,0.1,0.02)
        \helium(\x+0.5 ,\y+1.45,0.2, 0.7,0.1, 0.02)
        \helium(\x+1.1 ,\y+0.7, 0.3, 0.7,0.1, 0.02)

        % edge of each box
        \path[draw,thick] (\x,\y) rectangle (\x+2,\y+2);
    }
}

% central box thicker than other
\path[draw,ultra thick] (7,3) rectangle (9,5);

% an arrow
\path[draw,ultra thick,->] (2.5,4) -- (4.5,4);

% dashed lines
\foreach \x in {5,7,9,11} {
    % vertical
    \path[draw,dashed,thick] (\x,0) -- (\x,1) (\x,7) -- (\x,8);
    % horizontal
    \path[draw,dashed,thick] (4,\x-4) -- (5,\x-4) (11,\x-4) -- (12,\x-4);
}

\end{tikzpicture}