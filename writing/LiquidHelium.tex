\documentclass[Thesis.tex]{subfiles}
\begin{document}
\chapter{Liquid $^4$He}
\label{chp:liquid-helium}

We turn now to the much more challenging system of liquid Helium, presented
in~\cref{sec:liquid-helium-theory}. As before, we will first present the
benchmark result followed by the neural networks.


\section{Benchmark}

We will use one of the simpler benchmark wave functions for this system, known
as the McMillan form wave function~\cite{McMillan-1965}:

\begin{align}
  \label{eq:McMillan-wave-function-def}
  \psi_{M} &= \exp(-\frac{1}{2}\sum_{i<j} \qty(\frac{\beta}{r_{ij}})^5).
\end{align}
An important observation now is the lack of any single particle wave
function factor. In the case of the Quantum Dot we had a Gaussian localized at
the origin as a result of the potential well. This system, however, is infinite
and periodic without any such influence driving it towards particular points in
space. Furthermore, because of the lack of an external field the single particle
solutions are just free particles, and does not help us understand the many-body
system.

\subsection{Finite Size Dependency}

An important aspect of all the results that we will present is that they are
highly dependent on the number of particles used in the simulation box, as well
as the size of the box it self. We will hold the number density of particles
constant, $\rho$, and set the side lengths of the simulation box, $L$, depending on
the number of particles $N$:

\begin{align}
  L = \sqrt[3]{\frac{N}{\rho}}.
\end{align}

\noindent As the assumption of periodicity is a simplifying approximation, we
introduce some erroneous effects because of it. These generally disappear as we
increase the number of particles (and hence the size of box), but the
computation time needed to run the simulations increase significantly with
increasing numbers. The purpose of the following analysis is to test the
\emph{relative} accuracy of different wave functions. With that in mind we have
used a small number of particles in the main results, where the absolute error
introduced is significant. The number should hopefully still be large enough to
introduce all the relevant effects and produce valid test results.


\subsection{Optimizing}

We have optimized~\cref{eq:McMillan-wave-function-def} using $\num{5000}$
iterations of $\num{1000}$ MC cycles each. We used standard Metropolis sampling
along with the ADAM optimizer. \cref{fig:He-benchmark-training} shows the
progression of both the energy and variational parameter during training.
Because of the strong correlations involved, there is significantly more
variance in these results compared to the benchmark used for Quantum Dots. We
see the value for $\beta$ oscillating without any indication of converging to a
fixed value.

\cref{tab:He-benchmark-results} show the energy estimates from the final model.
The same wave function (i.e. same value for $\beta$) has also been used on
different number of particles to illustrate how the energy increases for larger
systems. Based on computations with larger and larger systems, the value for $N
= 256$ is quite close to the apparent convergence for $N\to\infty$.

\begin{figure}[h]
  \centering
  % This file was created by matplotlib2tikz v0.7.4.
\begin{tikzpicture}

\definecolor{color0}{rgb}{0.12156862745098,0.466666666666667,0.705882352941177}

\begin{groupplot}[group style={group size=2 by 1}]
\nextgroupplot[
tick align=outside,
tick pos=left,
x grid style={white!69.01960784313725!black},
xlabel={\% of training},
xmin=-5, xmax=105,
xtick style={color=black},
y grid style={white!69.01960784313725!black},
ylabel={Ground state energy (a.u.)},
ymin=-0.0267031755186082, ymax=-0.0259300855485359,
ytick style={color=black}
]
\addplot [semithick, color0]
table {%
0 -0.0260355588225757
1 -0.0262748291729487
2 -0.0261401291900171
3 -0.0261285730065402
4 -0.026519375347024
5 -0.0261085089752255
6 -0.026322018991364
7 -0.0261077897719258
8 -0.0262271509558243
9 -0.0263378611844199
10 -0.0264435164704036
11 -0.0260468974446812
12 -0.0261182326024303
13 -0.0264756681048654
14 -0.0260829755851611
15 -0.0262067425140259
16 -0.0259652260017211
17 -0.0261142101566383
18 -0.0261623463567444
19 -0.0262576609292962
20 -0.0265326433148494
21 -0.0265189212881611
22 -0.0263627505674447
23 -0.0263793057554275
24 -0.0264571056349538
25 -0.0261491150092845
26 -0.0263697738716649
27 -0.0260495198123338
28 -0.0262785115452796
29 -0.0263918091399123
30 -0.0261891024998046
31 -0.026182282409602
32 -0.0265774006198488
33 -0.0264442737818247
34 -0.0262053783253174
35 -0.0262980796722677
36 -0.026161471121124
37 -0.0264718910054234
38 -0.0264612475991768
39 -0.0263653945334238
40 -0.0266680350654231
41 -0.0265300262362027
42 -0.0262036089568738
43 -0.0264818664075457
44 -0.0263708805496209
45 -0.0263821604452956
46 -0.0262053500142723
47 -0.0264805122875391
48 -0.0262599853308704
49 -0.0264855746477873
50 -0.0262205772124631
51 -0.0263289145740529
52 -0.0264166794498659
53 -0.0263943726919734
54 -0.0264438531507356
55 -0.0264133019659749
56 -0.0261675058137356
57 -0.0265714049860063
58 -0.0263013144940345
59 -0.026526734559157
60 -0.0262162180142656
61 -0.0262366338651392
62 -0.0263139905784887
63 -0.0263707347216213
64 -0.0261434885201148
65 -0.0260564790789484
66 -0.0263806758247908
67 -0.0261465160940166
68 -0.0265138259748955
69 -0.0263950197835461
70 -0.0264490413820363
71 -0.026416617207186
72 -0.026084753813393
73 -0.0263018901085476
74 -0.0265211386978879
75 -0.0264420293985281
76 -0.0262121159078878
77 -0.0263659048883015
78 -0.0264159812700511
79 -0.0264125348077408
80 -0.0262545119198371
81 -0.0265789855566316
82 -0.0263928270344087
83 -0.0264235462537862
84 -0.0260584267486833
85 -0.0263308020892256
86 -0.0262857641316715
87 -0.0263176011200531
88 -0.0262221273019533
89 -0.0263796981979262
90 -0.0261900239319076
91 -0.026442581170448
92 -0.0264976185673772
93 -0.0262758363545835
94 -0.0262645430478063
95 -0.0263848116823473
96 -0.0263783513560127
97 -0.0262433736510786
98 -0.0265103363409152
99 -0.0263601595181734
100 -0.02636108196449
};

\nextgroupplot[
legend cell align={left},
legend style={at={(0.03,0.97)}, anchor=north west, draw=white!80.0!black},
tick align=outside,
tick pos=left,
x grid style={white!69.01960784313725!black},
xlabel={\% of training},
xmin=-5, xmax=105,
xtick style={color=black},
y grid style={white!69.01960784313725!black},
ymin=2.94829180504247, ymax=2.98587209410819,
ytick style={color=black}
]
\addplot [semithick, color0]
table {%
0 2.95
1 2.95103580692781
2 2.95305279010816
3 2.95460894023395
4 2.95577028904971
5 2.95650316293398
6 2.95825120836466
7 2.95935034209054
8 2.96133335776313
9 2.96383900347475
10 2.96448939120327
11 2.96542215358314
12 2.96711278853889
13 2.96910227479993
14 2.97109897692918
15 2.97224830715682
16 2.97294547465019
17 2.97448506512391
18 2.97419977930748
19 2.97461544110386
20 2.97474006216257
21 2.97530178911989
22 2.97683661217641
23 2.97677090335281
24 2.97762363936926
25 2.97743580399733
26 2.97870980652766
27 2.9784990859833
28 2.97850128657727
29 2.97939939301628
30 2.97997820064564
31 2.97805499792723
32 2.97734042969777
33 2.97785731987766
34 2.97733175806347
35 2.97678106695203
36 2.97637650864398
37 2.9778035160173
38 2.97905520632272
39 2.9784809989721
40 2.97767384035216
41 2.978984929289
42 2.98129434345163
43 2.98129455409872
44 2.98171724963203
45 2.98139119500532
46 2.98125509883751
47 2.98255231660043
48 2.98337149648428
49 2.98416389915065
50 2.98346559728962
51 2.98330470992909
52 2.98216887429106
53 2.98126880712631
54 2.98138781269188
55 2.98208984196603
56 2.98117487047914
57 2.9809486172087
58 2.98133319898255
59 2.9820499975273
60 2.98207107235994
61 2.98268141802709
62 2.98376332654432
63 2.9820082804167
64 2.98109941224646
65 2.98062306185773
66 2.98102920941746
67 2.98193484440834
68 2.98330259012095
69 2.98270627077085
70 2.98327264397638
71 2.98338885422558
72 2.98265491386629
73 2.98265603599819
74 2.9821677681507
75 2.98300324304189
76 2.98147512692259
77 2.98014500448989
78 2.97966424676313
79 2.97989478702793
80 2.97913891125796
81 2.97998144813434
82 2.98000257643695
83 2.98016945501647
84 2.98116795619149
85 2.98101613035288
86 2.98037557115624
87 2.98121680812499
88 2.98197274591324
89 2.98056398797002
90 2.98007918335043
91 2.97999522453107
92 2.97902301471611
93 2.97970203870668
94 2.9797799620843
95 2.97981771051079
96 2.97997725307989
97 2.98022147456089
98 2.97992843995694
99 2.98032522120503
100 2.98069580889328
};
\addlegendentry{$\beta$}
\end{groupplot}

\end{tikzpicture}
  \caption{Helium benchmark training.}
  \label{fig:He-benchmark-training}
\end{figure}

\begin{table}[h]
  \centering
  \begin{tabular}{lS[table-format=3.4]*2{S[table-format=3.2]}*2{S[table-format=2.1]}}
\toprule
\addlinespace
& {$\langle E_L\rangle$} & {CI$^{95}_-$} & {CI$^{95}_+$} & {Std} & {Var} \\
\addlinespace
\midrule
\addlinespace
\addlinespace
    $\psi_M^{(32)}$ & -6.82(3) & -6.88 & -6.76 & \num{4.6e-01} & \num{2.1e-01}\\
$\psi_M^{(64)}$ & -6.13(3) & -6.18 & -6.07 & \num{4.2e-01} & \num{1.8e-01}\\
$\psi_M^{(256)}$ & -5.79(2) & -5.83 & -5.75 & \num{2.3e-01} & \num{5.2e-02}\\
\addlinespace\addlinespace\bottomrule
\end{tabular}
  \caption{Helium benchmark results}
  \label{tab:He-benchmark-results}
\end{table}


\begin{itemize}
\item Show plot of training for 32 particles
\item In table of energies, include same wave function on larger numbers (64,
  128, 256)
\end{itemize}


\end{document}
