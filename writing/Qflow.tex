\documentclass[Thesis.tex]{subfiles}
\begin{document}
\chapter{QFLOW: Software Package for VMC with Neural Networks}
\label{chp:qflow}

Everything discussed so far has culminated into the development of a self
contained package available to the public under the permissive MIT license. All
original results presented in the next part are produced with this package, and
the source code to reproduce them are also presented in all cases.

This chapter is dedicated to the presentation of the package, explanations of
design decisions and comparison with other tools. A more complete technical
documentation is published online~\cite{qflow}, and it is recommended to consult this
resource for practical usage.

\section{Installation}

Refer to~\cite{qflow} for up to date instructions. Here we only mention the
required components for using QFLOW:

\begin{itemize}
  \item Python 3.7 or greater
  \item A C++17 compliant compiler
  \item OpenMPI 3.1.3 or greater (or equivalent MPI implementations)
\end{itemize}
Every other dependency should be automatically resolved during installation.

\section{Quickstart}

The following example aims to showcase the basic usage pattern of QFLOW.

Say we want run a VMC optimization of two electrons in two
dimensions confined in a harmonic oscillator trap, i.e. the quantum dot system
described in \cref{sec:quantum-dots-theory}. The following complete code runs an
optimization using a Pade-Jastrow wave function and evaluates the result. All in
a few readable lines of Python:\\

\lstinputlisting[language=Python, lastline=42, basicstyle=\scriptsize]{scripts/quickstart.py}

\begin{lstlisting}[basicstyle=\scriptsize]
  OUTPUT:
      {'CI':  (3.0005518930024406, 3.000882266251301),
      'max':  3.0203576916357706,
      'mean': 3.0007170796268707,
      'min':  2.9820746671373106,
      'sem':  8.425552793020659e-05,
      'std':  0.005391695501040196,
      'var':  2.9070380375937085e-05}
\end{lstlisting}

\begin{figure}[h]
  \centering
  % \resizebox{0.7\linewidth}{!}{%
      % This file was created by matplotlib2tikz v0.7.4.
\begin{tikzpicture}

\definecolor{color0}{rgb}{0.12156862745098,0.466666666666667,0.705882352941177}
\definecolor{color1}{rgb}{1,0.498039215686275,0.0549019607843137}
\definecolor{color2}{rgb}{0.172549019607843,0.627450980392157,0.172549019607843}

\begin{groupplot}[group style={group size=2 by 1}]
\nextgroupplot[
tick align=outside,
tick pos=left,
title={$\langle E_L\rangle$ [a.u]},
x grid style={white!69.01960784313725!black},
xmin=-5, xmax=105,
xtick style={color=black},
y grid style={white!69.01960784313725!black},
ymin=2.97679626711076, ymax=3.48370206437361,
ytick style={color=black}
]
\addplot [semithick, color0]
table {%
0 3.46066089177075
1 3.05697702464692
2 3.0271684085862
3 3.02218275734814
4 3.03021959831227
5 3.02561687477614
6 3.02627411275781
7 3.02516281006362
8 3.02389528964726
9 3.01960300479133
10 3.02196222032015
11 3.01493911652559
12 3.0238953522491
13 3.01333243894374
14 3.0183587443204
15 3.01488696837594
16 3.01722300283271
17 3.01257451248019
18 3.01319988320326
19 3.01284972658477
20 3.01138335514942
21 3.01295800190302
22 3.01402571542143
23 3.0121864685687
24 3.00805759367618
25 3.008777184152
26 3.0095415688957
27 3.01126951899633
28 3.00647814684431
29 3.00548609904375
30 3.00598127356645
31 3.00640625160984
32 3.00549831194315
33 3.00410730849279
34 3.00435765218153
35 3.00579955760118
36 3.00657302258194
37 3.00453066444318
38 3.00461705164295
39 3.0034046083838
40 3.0044614684543
41 3.00423273718025
42 3.00134379470422
43 3.00183122506268
44 3.00311165776617
45 3.00283347966883
46 3.00075541559281
47 3.00053325001442
48 3.00081362268715
49 3.00011026006052
50 3.00192711054592
51 3.00069022966182
52 3.00237613979508
53 3.00248128340816
54 3.00154827514289
55 3.00123302173879
56 3.00071241326163
57 3.00066401445506
58 3.00067498259921
59 3.00075581025347
60 3.00057240265175
61 3.00141347130951
62 3.00069819279257
63 3.00149501695416
64 3.00115623127114
65 3.00126995886738
66 3.00010893505924
67 3.00059844067248
68 3.00094186268187
69 2.99983743971361
70 3.0007698820108
71 3.00022851003444
72 3.00046096693956
73 3.00143052309649
74 3.00150173926704
75 3.00114828314348
76 3.00032271331063
77 3.00074711079175
78 3.00030707783872
79 3.00073354470591
80 3.00033167164613
81 3.00088304715914
82 3.00124275154125
83 3.00079708820991
84 3.00066387860218
85 3.0007771930537
86 3.00064073365758
87 3.00009539020464
88 3.00092500617943
89 3.00025364340319
90 3.00083362535462
91 3.00054380746313
92 3.00030983333914
93 3.00113066134316
94 3.00045527403702
95 3.00048444265464
96 3.00028339069915
97 3.00031728018598
98 3.0004816172366
99 3.00030048221465
100 3.00122058807899
};

\nextgroupplot[
legend cell align={left},
legend style={at={(0.91,0.5)}, anchor=east, draw=white!80.0!black},
tick align=outside,
tick pos=left,
title={Parameters},
x grid style={white!69.01960784313725!black},
xmin=-5, xmax=105,
xtick style={color=black},
y grid style={white!69.01960784313725!black},
ymin=0.378191221816561, ymax=1.02960994181826,
ytick style={color=black}
]
\addplot [semithick, color0]
table {%
0 0.8
1 0.508680575886114
2 0.454184658420057
3 0.435688427614269
4 0.444298314220426
5 0.438225775135671
6 0.428520103241169
7 0.456501178240051
8 0.434565065579558
9 0.445417230936385
10 0.444909357011198
11 0.435347082203193
12 0.463736821680238
13 0.449397920805592
14 0.439118770672357
15 0.439105848270841
16 0.436430822783328
17 0.450974263933736
18 0.443315316119028
19 0.453728339208235
20 0.43953425995884
21 0.451905608377557
22 0.447446775157672
23 0.442551891183911
24 0.46347533928939
25 0.455439197570969
26 0.452093874958556
27 0.452335600341889
28 0.464419667477833
29 0.464804511809617
30 0.452056757948541
31 0.461468027899692
32 0.455302134215792
33 0.459599682733049
34 0.467394608849733
35 0.463198903451213
36 0.470240410943616
37 0.47407255252146
38 0.463732456447168
39 0.471510148929472
40 0.465274542379048
41 0.476330697662877
42 0.477729617975402
43 0.477112563306
44 0.471635092031067
45 0.472870399202977
46 0.481726040073444
47 0.48257200636387
48 0.480266023797381
49 0.477206800041736
50 0.480240102720675
51 0.481189410980784
52 0.481425435049959
53 0.482301048128581
54 0.482302526376746
55 0.486249907462193
56 0.481850107330233
57 0.484521777214977
58 0.480978249934775
59 0.488473974359684
60 0.486776202242288
61 0.486274576730671
62 0.491039499034024
63 0.482296712691818
64 0.494888148403905
65 0.490942542852218
66 0.487152427334319
67 0.488036020306612
68 0.488362878094785
69 0.489466148596936
70 0.489242365041852
71 0.487286908342893
72 0.488389175568377
73 0.485918442867768
74 0.488332971278629
75 0.489660119367009
76 0.491233407917741
77 0.491672884066782
78 0.487252127952925
79 0.489748222546138
80 0.487840126247182
81 0.490446995437946
82 0.486814079695444
83 0.489827626347833
84 0.493476056395488
85 0.48522704260437
86 0.491173556375783
87 0.489937997363697
88 0.488942755356857
89 0.488670525124645
90 0.490988345568276
91 0.493169632830697
92 0.488380386149122
93 0.4899686157723
94 0.489280312707222
95 0.493633129823782
96 0.493277132058139
97 0.491721029906923
98 0.493007994171152
99 0.491430789331709
100 0.489845062449944
};
\addlegendentry{Gaussian alpha}
\addplot [semithick, color1]
table {%
0 1
1 1
2 1
3 1
4 1
5 1
6 1
7 1
8 1
9 1
10 1
11 1
12 1
13 1
14 1
15 1
16 1
17 1
18 1
19 1
20 1
21 1
22 1
23 1
24 1
25 1
26 1
27 1
28 1
29 1
30 1
31 1
32 1
33 1
34 1
35 1
36 1
37 1
38 1
39 1
40 1
41 1
42 1
43 1
44 1
45 1
46 1
47 1
48 1
49 1
50 1
51 1
52 1
53 1
54 1
55 1
56 1
57 1
58 1
59 1
60 1
61 1
62 1
63 1
64 1
65 1
66 1
67 1
68 1
69 1
70 1
71 1
72 1
73 1
74 1
75 1
76 1
77 1
78 1
79 1
80 1
81 1
82 1
83 1
84 1
85 1
86 1
87 1
88 1
89 1
90 1
91 1
92 1
93 1
94 1
95 1
96 1
97 1
98 1
99 1
100 1
};
\addlegendentry{Jastrow Alpha}
\addplot [semithick, color2]
table {%
0 1
1 0.965247156991024
2 0.949160024892387
3 0.93389213064639
4 0.923595053303821
5 0.912153057972033
6 0.899309869033163
7 0.887278207322025
8 0.873596920860959
9 0.860168132114827
10 0.845919551176484
11 0.832806288631121
12 0.819983954582874
13 0.80735511696165
14 0.794478594769299
15 0.782894655210917
16 0.771082825239821
17 0.759206135043222
18 0.747604213038102
19 0.73797264995329
20 0.725737631376208
21 0.708647062238535
22 0.699134739127615
23 0.687739008894988
24 0.675050057414953
25 0.663980595364123
26 0.652251712524369
27 0.643058509769019
28 0.632867529171962
29 0.6239130376108
30 0.612891916512401
31 0.603456196582262
32 0.595157432434952
33 0.587022793934797
34 0.578797347213793
35 0.570167201587309
36 0.562880594740422
37 0.556602161017858
38 0.547289348790961
39 0.540583427920133
40 0.532640761357728
41 0.526044596186346
42 0.519126365350005
43 0.511293724558068
44 0.502730733800298
45 0.498019407514713
46 0.493152044526496
47 0.486028759453034
48 0.479146981365096
49 0.47229704997195
50 0.467459180298608
51 0.465275516144942
52 0.463210714080641
53 0.460212111400731
54 0.458078070602196
55 0.456198650276278
56 0.453568395527642
57 0.452394049730842
58 0.449933497076148
59 0.448836983682311
60 0.446910678380825
61 0.445140208330659
62 0.444037788301431
63 0.440803948120464
64 0.439930367686976
65 0.437299642592568
66 0.435589147586754
67 0.434234686634496
68 0.433253173515987
69 0.431723944357054
70 0.430715934693467
71 0.428502396618651
72 0.426833331518134
73 0.425768778523464
74 0.424761925144455
75 0.423385767150938
76 0.423173989442718
77 0.422476863386904
78 0.420776944487903
79 0.420713433951991
80 0.41984892414052
81 0.419781147515997
82 0.418541586191666
83 0.418313582147631
84 0.418011034440882
85 0.415096386642692
86 0.415642420920087
87 0.414536534795712
88 0.413917198983777
89 0.412888861859648
90 0.412986985168265
91 0.41309980620354
92 0.411249709879786
93 0.410419647109191
94 0.409658897551261
95 0.410828215957129
96 0.410656955921689
97 0.409703449557808
98 0.409886448697207
99 0.408994846693284
100 0.40780116363482
};
\addlegendentry{Jastrow Beta}
\end{groupplot}

\end{tikzpicture}
  % }
  \caption{Learning progression in terms of energy (left) and variational
    parameters (right). The source code for this graphic can be found
    at ~\cite[TODO: Add
    path]{MS-thesis-repository}, and \LaTeX{} output generated
    by~\cite{nico_schlomer_2018_1173090}}
  \label{fig:quickstart-example}
\end{figure}

\noindent The script can be run under MPI (\texttt{mpiexec -n X python script.py}) with any number of processes, which
results in almost completely linear speedup.

\section{Structure}

The package has been structured around four basic building blocks of a VMC
calculation, each of which is represented by a base class:

\begin{itemize}
\item Hamiltonians
\item Wave functions
\item Samplers
\item Optimizers
\end{itemize}

Due to this object oriented design approach, each part needs only consider how
to interact with the base classes of the other three, as opposed to duplicating
code per combination. As well as aiding development, this makes it easy the
prototype various combinations of wave functions, sampling strategies etc., as
well as trivial to exchange the Hamiltonian in question.


\subsection{Hamiltonians}

The \texttt{qflow.\-hamiltonians.\-Hamiltonian} class in QFLOW is responsible for defining the energy
(kinetic, external potentials and interaction potentials) of the system. This
class is queried for local energy evaluations/gradients as well other system
related quantities of interest. Each particular Hamiltonian is implemented as a
subclass of \texttt{qflow.\-hamiltonians.\-Hamiltonian}, and at a minimum it needs to define the
external and internal potentials (can be set equal to zero).

\subsection{Wave functions}

In a similar way, the \texttt{qflow.\-wavefunctions.\-Wavefunction} class in QFLOW defines the basic
operations required for arbitrary wave functions, and each particular trial wave
function is implemented as a subclass to this.

\subsubsection{Neural Networks}

The neural networks are used to define wave functions, and as such their
relevant definitions are structured under the \texttt{qflow.\-wavefunction} module. The
class \texttt{qflow.\-wavefunctions.\-Dnn} (Deep Neural Network) is a subclass of \texttt{qflow.\-wavefunctions.\-Wavefunction}.
Setting up the network it self is done by iteratively adding layer objects (see
later examples).

\subsection{Samplers}

Monte Carlo sampling is done through one of the implemented methods in the
\texttt{qflow.samplers} module. Again the code is structured around the base
class \texttt{qflow.samplers.Sampler}, which defines the basic interface to the
sampler, including querying for samples (system configurations) and obtaining
the acceptance rate. Specific algorithms are subclasses, such as
\texttt{qflow.\-samplers.\-MetropolisSampler} and
\texttt{qflow.\-samplers.\-ImportanceSampler}.


\subsection{Optimizers}

The last major piece is the \texttt{qflow.optimizers} module. The base class
here is \texttt{qflow.\-optimizers.\-SgdOptimzier} (Stochastic Gradient Decent),
which is the vanilla implementation from
\cref{eq:fixed-learning-rate-update-rule}. Similarly we have
\texttt{qflow.\-optimizers.\-AdamOptimizer} as a subclass of the former, which
unsurprisingly implements \cref{eq:adam-default-parameters}.


\section{Inheritance vs. Templates}

C++ has a powerful templating feature which allows us to generate code for
particular types on demand. We could have used this instead of inheritance in
order to compose Hamiltonians, wave functions etc., without the extra overhead
and bloat often associated with inheritance. While we acknowledge that a pure
template implementation would potentially give a speed increase, the inheritance
method was chosen for the following reasons:

\begin{itemize}
\item The speed increase would likely be small.
\item Compilation time would increase significantly, as unique template
  instantiations would be required for each combination
\item Selecting a different set of Hamiltonians, wave functions, samplers and
  optimizers would require another compilation run
\item A Python interface would not be possible. In order to generate the
  Python bindings we need a pre-built C++ library.
\end{itemize}


\end{document}
