\documentclass[Thesis.tex]{subfiles}
\begin{document}
\chapter{Introduction}
\label{chp:introduction}

Motivating \emph{why} we want to do this...


\section{How to Use This Thesis}

\subsection{Structure}
\cref{prt:theory} starts with a introduction to all the underlying theory
necessary to understand the results of this work. \cref{chp:the-quantum-problem}
starts with introducing the relevant bits of Quantum Mechanics, followed
immediately by a presentation of Variational Monte Carlo and Monte Carlo methods
in general (\cref{chp:variational-monte-carlo,chp:monte-carlo}). This forms the
backbone of the thesis. Next we shift gears and discuss Machine Learning in
general (\cref{chp:machine-learning}), focusing again on the parts which are
most relevant to us. We end \cref{prt:theory} by talking about the glue that
connects all of this together (\cref{chp:mergin-vmc-with-ml}).

\cref{prt:implementation} is all about the technical details of how we implement
the algorithms from \cref{prt:theory} into efficient and correct code. We
discuss design choices (\cref{chp:parallelization,chp:auto-diff}), and culminate
with presenting QFLOW, the library we have
developed for all our computing needs (\cref{chp:qflow}). Lastly we make an effort to convince the
reader that the code is correct by verifying the results on some
selected idealized cases (\cref{chp:verfication}).

\cref{prt:results} finally presents the results of the new method we have
developed, testing it on both a few-body system (\cref{chp:quantum-dots}) and a
more complicated many-body system (\cref{chp:liquid-helium}). 

Finally, \cref{prt:conclusion} offers conclusions and future prospects.

\subsection{Reproducible}

One of the most frustrating parts of writing this thesis has been attempting to
reproduce results from published articles. Far to often vital details are left
out, both in theory and implementation, leaving poor soles to guess as to how
they achieved the results they present.

We have made a conscious effort to be
better in this regard. To that end, all of the code, data, figures etc.\ that
you find in this thesis are openly available at one central
repository~\cite{qflow}. That even includes the source code for this very
document. The QFLOW library is permissively licensed under the MIT license,
allowing anyone to use the code how ever they see fit. Furthermore, \emph{every} table
and figure has a reference to where you may find the exact source code which
generated it, leaving no doubt as to the details of any results.

\subsection{Notation}

Another point of possible frustration is unclear use of notation. While we
assume a certain general familiarity with mathematics, we strive to make our
notation as clear as possible. All symbols should an accompanying
explanation following their first use. Furthermore, while we attempt to use
standard notation where possible, any doubt should be removed by consulting the
notation reference in \cref{app:notation-reference}.

\end{document}
